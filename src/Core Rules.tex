\part{Core Rules}\index{Core Rules}

\chapter{Introduction}\index{Core Rules!Introduction}

\begin{multicols}{2}

\section{How to Play the Cypher System}\index{Core Rules!Introduction!How to Play Cypher System}
The rules of the Cypher System are quite straightforward at their heart, as all of gameplay is based around a few core concepts.

This chapter provides a brief explanation of how to play the game, and it’s useful for learning the game. Once you understand the basic concepts, you’ll likely want to reference Rules of the Game for a more in-depth treatment. 

The Cypher System uses a twenty-sided die (1d20) to determine the results of most actions. Whenever a roll of any kind is called for and no die is specified, roll a d20.

The game master sets a difficulty for any given task. There are ten degrees of difficulty. Thus, the difficulty of a task can be rated on a scale of 1 to 10.

Each difficulty has a target number associated with it. The target number is always three times the task’s difficulty, so a difficulty 1 task has a target number of 3, but a difficulty 4 task has a target number of 12. To succeed at the task, you must roll the target number or higher. See the Task Difficulty table for guidance in how this works.

Character skills, favorable circumstances, or excellent equipment can decrease the difficulty of a task. For example, if a character is trained in climbing, they turn a difficulty 6 climb into a difficulty 5 climb. This is called easing the difficulty by one step (or just easing the difficulty, which assumes it’s eased by one step). If they are specialized in climbing, they turn a difficulty 6 climb into a difficulty 4 climb. This is called easing the difficulty by two steps. Decreasing the difficulty of a task can also be called easing a task. Some situations increase, or hinder, the difficulty of a task. If a task is hindered, it increases the difficulty by one step.

A skill is a category of knowledge, ability, or activity relating to a task, such as climbing, geography, or persuasiveness. A character who has a skill is better at completing related tasks than a character who lacks the skill. A character’s level of skill is either trained (reasonably skilled) or specialized (very skilled).
If you are trained in a skill relating to a task, you ease the difficulty of that task by one step. If you are specialized, you ease the difficulty by two steps. A skill can never decrease a task’s difficulty by more than two steps.

Anything else that reduces difficulty (help from an ally, a particular piece of equipment, or some other advantage) is referred to as an asset. Assets can never decrease a task’s difficulty by more than two steps.

You can also decrease the difficulty of a given task by applying Effort. (Effort is described in more detail in the Rules of the Game chapter.) 
To sum up, three things can decrease a task’s difficulty: skills, assets, and Effort.

If you can ease a task so its difficulty is reduced to 0, you automatically succeed and don’t need to make a roll.

\subsection{When Do You Roll?}\index{Core Rules!Introduction!When Do You Roll?} 

Any time your character attempts a task, the GM assigns a difficulty to that task, and you roll a d20 against the associated target number.

When you jump from a burning vehicle, swing an axe at a mutant beast, swim across a raging river, identify a strange device, convince a merchant to give you a lower price, craft an object, use a power to control a foe’s mind, or use a blaster rifle to carve a hole in a wall, you make a d20 roll.

However, if you attempt something that has a difficulty of 0, no roll is needed—you automatically succeed. Many actions have a difficulty of 0. Examples include walking across the room and opening a door, using a special ability to negate gravity so you can fly, using an ability to protect your friend from radiation, or activating a device (that you already understand) to erect a force field. These are all routine actions and don’t require rolls.

Using skill, assets, and Effort, you can ease the difficulty of potentially any task to 0 and thus negate the need for a roll. Walking across a narrow wooden beam is tricky for most people, but for an experienced gymnast, it’s routine. You can even ease the difficulty of an attack on a foe to 0 and succeed without rolling.

If there’s no roll, there’s no chance for failure. However, there’s also no chance for remarkable success (in the Cypher System, that usually means rolling a 19 or 20, which are called special rolls; the Rules of the Game chapter also discusses special rolls).

\subsection{Task Difficulty}

\end{multicols}

\begin{table}
\centering

\caption{Task Difficulty}
\label{tab:Task Difficulty}

\begin{tabularx}{\textwidth}{| p{0.15\textwidth} | p{0.3\textwidth} | p{0.15\textwidth} | p{0.3\textwidth} |}
\hline
\textbf{Task \newline Difficulty} & \textbf{Description} & \textbf{Target Number} & \textbf{Guidance} \\
\hline
% Tutaj wpisujesz dane do tabeli
0 & Routine & 0 & Anyone can do this basically every time.  \\ \hline
1 & Simple & 3 & Most people can do this most of the time.  \\ \hline
2 & Standard & 6 & Typical task requiring focus, but most people can usually do this.  \\ \hline
3 & Demanding & 9 & Requires full attention; most people have a 50/50 chance to succeed.  \\ \hline
4 & Difficult & 12 & Trained people have a 50/50 chance to succeed.  \\ \hline
5 & Challenging & 15 & Even trained people often fail.  \\ \hline
6 & Intimidating & 18 & Normal people almost never succeed.  \\ \hline
7 & Formidable & 21 & Impossible without skills or great effort.  \\ \hline
8 & Heroic & 24 & A task worthy of tales told for years afterward.  \\ \hline
9 & Immortal & 27 & A task worthy of legends that last lifetimes.  \\ \hline
10 & Impossible & 30 & A task that normal humans couldn’t consider (but one that doesn’t break the laws of physics).  \\ \hline

\end{tabularx}

\end{table}

\begin{multicols}{2}

\subsection{Combat}\index{Core Rules!Introduction!Combat}

Making an attack in combat works the same way as any other roll: the GM assigns a difficulty to the task, and you roll a d20 against the associated target number.

The difficulty of your attack roll depends on how powerful your opponent is. Just as tasks have a difficulty from 1 to 10, creatures have a level from 1 to 10. Most of the time, the difficulty of your attack roll is the same as the creature’s level. For example, if you attack a level 2 bandit, it’s a level 2 task, so your target number is 6.

It’s worth noting that players make all die rolls. If a character attacks a creature, the player makes an attack roll. If a creature attacks a character, the player makes a defense roll.

The damage dealt by an attack is not determined by a roll—it’s a flat number based on the weapon or attack used. For example, a spear always does 4 points of damage.

Your Armor characteristic reduces the damage you take from attacks directed at you. You get Armor from wearing physical armor (such as a leather jacket in a modern game or chainmail in a fantasy setting) or from special abilities. Like weapon damage, Armor is a flat number, not a roll. If you’re attacked, subtract your Armor from the damage you take. For example, a leather jacket gives you +1 to Armor, meaning that you take 1 less point of damage from attacks. If a mugger hits you with a knife for 2 points of damage while you’re wearing a leather jacket, you take only 1 point of damage. If your Armor reduces the damage from an attack to 0, you take no damage from that attack.

When you see the word “Armor” capitalized in the game rules (other than in the name of a special ability), it refers to your Armor characteristic—the number you subtract from incoming damage. When you see the word “armor” with a lowercase “a,” it refers to any physical armor you might wear.
Typical physical weapons come in three categories: light, medium and heavy.

Light weapons inflict only 2 points of damage, but they ease attack rolls because they are fast and easy to use. Light weapons are punches, kicks, clubs, knives, handaxes, rapiers, small pistols, and so on. Weapons that are particularly small are light weapons.

Medium weapons inflict 4 points of damage. Medium weapons include swords, battleaxes, maces, crossbows, spears, pistols, blasters, and so on. Most weapons are medium. Anything that could be used in one hand (even if it’s often used in two hands, such as a quarterstaff or spear) is a medium weapon.

Heavy weapons inflict 6 points of damage, and you must use two hands to attack with them. Heavy weapons are huge swords, great hammers, massive axes, halberds, heavy crossbows, blaster rifles, and so on. Anything that must be used in two hands is a heavy weapon.

\subsection{Special Rolls}\index{Core Rules!Introduction!Special Rolls}

When you roll a natural 19 (the d20 shows “19”) and the roll is a success, you also have a minor effect. In combat, a minor effect inflicts 3 additional points of damage with your attack, or, if you’d prefer a special result, you could decide instead that you knock the foe back, distract them, or something similar. When not in combat, a minor effect could mean that you perform the action with particular grace. For example, when jumping down from a ledge, you land smoothly on your feet, or when trying to persuade someone, you convince them that you’re smarter than you really are. In other words, you not only succeed but also go a bit further.

When you roll a natural 20 (the d20 shows “20”) and the roll is a success, you also have a major effect. This is similar to a minor effect, but the results are more remarkable. In combat, a major effect inflicts 4 additional points of damage with your attack, but again, you can choose instead to introduce a dramatic event such as knocking down your foe, stunning them, or taking an extra action. Outside of combat, a major effect means that something beneficial happens based on the circumstance. For example, when climbing up a cliff wall, you make the ascent twice as fast. When a roll grants you a major effect, you can choose to use a minor effect instead if you prefer.

In combat (and only in combat), if you roll a natural 17 or 18 on your attack roll, you add 1 or 2 additional points of damage, respectively. Neither roll has any special effect options—just the extra damage.

(For more information on special rolls and how they affect combat and other interactions, see Rules of the Game.)

Rolling a natural 1 is always bad. It means that the GM introduces a new complication into the encounter.

\subsection{Glossary}\index{Core Rules!Introduction!Glossary}

\textbf{Game master (GM)}: The player who doesn’t run a character, but instead guides the flow of the story and runs all the NPCs.

\textbf{Nonplayer character (NPC)}: Characters run by the GM. Think of them as the minor characters in the story, or the villains or opponents. This includes any kind of creature as well as people.

\textbf{Party}: A group of player characters (and perhaps some NPC allies).

\textbf{Player character (PC)}: A character run by a player rather than the GM. Think of the PCs as the main characters in the story.

\textbf{Player}: The players who run characters in the game.

\textbf{Session}: A single play experience. Usually lasts a few hours. Sometimes one adventure can be accomplished in a session. More often, one adventure is multiple sessions.

\textbf{Adventure}: A single portion of the campaign with a beginning and an end. Usually defined at the beginning by a goal put forth by the PCs and at the end by whether or not they achieve that goal.

\textbf{Campaign}: A series of sessions strung together with an overarching story (or linked stories) with the same player characters. Often, but not always, a campaign involves a number of adventures.

\textbf{Character}: Anything that can act in the game. Although this includes PCs and human NPCs, it also technically includes creatures, aliens, mutants, automatons, animate plants, and so on. The word “creature” is usually synonymous.

\subsection{Range and speed}\index{Core Rules!Introduction!Range and Speed}

Distance is simplified into four categories: immediate, short, long, and very long.

Immediate distance from a character is within reach or within a few steps. If a character stands in a small room, everything in the room is within immediate distance. At most, immediate distance is 10 feet (3 m).

Short distance is anything greater than immediate distance but less than 50 feet (15 m) or so. 

Long distance is anything greater than short distance but less than 100 feet (30 m) or so. 

Very long distance is anything greater than long distance but less than 500 feet (150 m) or so. Beyond that range, distances are always specified—1,000 feet (300 m), a mile (1.5 km), and so on.

The idea is that it’s not necessary to measure precise distances. Immediate distance is right there, practically next to the character. Short distance is nearby. Long distance is farther off. Very long distance is really far off.

All weapons and special abilities use these terms for ranges. For example, all melee weapons have immediate range—they are close-combat weapons, and you can use them to attack anyone within immediate distance. A thrown knife (and most other thrown weapons) has short range. A bow has long range. An Adept’s Onslaught ability also has short range.

A character can move an immediate distance as part of another action. In other words, they can take a few steps over to the control panel and activate a switch. They can lunge across a small room to attack a foe. They can open a door and step through.

A character can move a short distance as their entire action for a turn. They can also try to move a long distance as their entire action, but the player might have to roll to see if the character slips, trips, or stumbles as the result of moving so far so quickly.

For example, if the PCs are fighting a group of cultists, any character can likely attack any cultist in the general melee—they’re all within immediate range. Exact positions aren’t important. Creatures in a fight are always moving, shifting, and jostling, anyway. However, if one cultist stayed back to fire a pistol, a character might have to use their entire action to move the short distance required to attack that foe. It doesn’t matter if the cultist is 20 feet (6 m) or 40 feet (12 m) away—it’s simply considered short distance. It does matter if the cultist is more than 50 feet (15 m) away because that distance would require a long or very long move.

(Many rules in this system avoid the cumbersome need for precision. Does it really matter if the ghost is 13 feet away from you or 18? Probably not. That kind of needless specificity only slows things down and draws away from, rather than contributes to, the story.)

\subsection{Experience Points}\index{Core Rules!Introduction!Experience Points}

Experience points (XP) are rewards given to players when the GM intrudes on the story (this is called GM intrusion) with a new and unexpected challenge. For example, in the middle of combat, the GM might inform the player that they drop their weapon. However, to intrude in this manner, the GM must award the player 2 XP. The rewarded player, in turn, must immediately give one of those XP to another player and justify the gift (perhaps the other player had a good idea, told a funny joke, performed an action that saved a life, and so on).

Alternatively, the player can refuse the GM intrusion. If they do so, they don’t get the 2 XP from the GM, and they must also spend 1 XP that they already have. If the player has no XP to spend, they can’t refuse the intrusion.

The GM can also give players XP between sessions as a reward for making discoveries during an adventure. Discoveries are interesting facts, wondrous secrets, powerful artifacts, answers to mysteries, or solutions to problems (such as where the kidnappers are keeping their victim or how the PCs repair the starship). You don’t earn XP for killing foes or overcoming standard challenges in the course of play. Discovery is the soul of the Cypher System.

Experience points are used primarily for character advancement (for details, see the Creating Your Character chapter), but a player can also spend 1 XP to reroll any die roll and take the better of the two rolls.

\subsection{Cyphers}\index{Core Rules!Introduction!Cyphers}

Cyphers are abilities that have a single use. In many campaigns, cyphers aren’t physical objects—they might be a spell cast upon a character, a blessing from a god, or just a quirk of fate that gives them a momentary advantage. In some campaigns, cyphers are physical objects that characters can carry. Whether or not cyphers are physical objects, they are part of the character (like equipment or a special ability) and are things characters can use during the game. The form that physical cyphers take depends on the setting. In a fantasy world they might be wands or potions, but in a science fiction game they could be alien crystals or prototype devices.

Characters will find new cyphers frequently in the course of play, so players shouldn’t hesitate to use their cypher abilities. Because cyphers are always different, the characters will always have new special powers to try.

\subsection{Other Dice}\index{Core Rules!Introduction!Other Dice}

In addition to a d20, you’ll need a d6 (a six-sided die). Rarely, you’ll need to roll a number between 1 and 100 (often called a d100 or d\% roll), which you can do by rolling a d20 twice, using the last digit of the first roll as the “tens” place and the last digit of the second roll as the “ones” place. For example, rolling a 17 and a 9 gives you 79, rolling a 3 and an 18 gives you 38, and rolling a 20 and a 10 gives you 00 (also known as 100). If you have a d10 (a ten-sided die), you can use it instead of the d20 to roll numbers between 1 and 100.

(A d6 is used most often for recovery rolls and to determine the level of cyphers.

\subsection{Creating Your Character}\index{Core Rules!Introduction!Creating Your Character}

This section explains how to create characters to play in a Cypher System game. This involves a series of decisions that will shape your character, so the more you understand what kind of character you want to play, the easier character creation will be. The process involves understanding the values of three game statistics and choosing three aspects that determine your character’s capabilities.

\subsubsection{Character Stats}\index{Core Rules!Introduction!Character Stats}

Every player character has three defining characteristics, which are typically called “statistics” or “stats.” These stats are Might, Speed, and Intellect. They are broad categories that cover many different but related aspects of a character.

\paragraph{Might}\index{Core Rules!Introduction!Character Stats!Might}

Might defines how strong and durable your character is. The concepts of strength, endurance, constitution, hardiness, and physical prowess are all folded into this one stat. Might isn’t relative to size; instead, it’s an absolute measurement. An elephant has more Might than the mightiest tiger, which has more Might than the mightiest rat, which has more Might than the mightiest spider.

Might governs actions from forcing doors open to walking for days without food to resisting disease. It’s also the primary means of determining how much damage your character can sustain in a dangerous situation. Physical characters, tough characters, and characters interested in fighting should focus on Might.

(Might could be thought of as Might/Health because it governs how strong you are and how much physical punishment you can take.)

\paragraph{Speed}\index{Core Rules!Introduction!Character Stats!Speed}

Speed describes how fast and physically coordinated your character is. The stat embodies quickness, movement, dexterity, and reflexes. Speed governs such divergent actions as dodging attacks, sneaking around quietly, and throwing a ball accurately. It helps determine whether you can move farther on your turn. Nimble, fast, or sneaky characters will want good Speed stats, as will those interested in ranged combat.

(Speed could be thought of as Speed/Agility because it governs your overall swiftness and reflexes.)

\paragraph{Intelect}\index{Core Rules!Introduction!Character Stats!Intelect}

This stat determines how smart, knowledgeable, and likable your character is. It includes intelligence, wisdom, charisma, education, reasoning, wit, willpower, and charm. Intellect governs solving puzzles, remembering facts, telling convincing lies, and using mental powers. Characters interested in communicating effectively, being learned scholars, or wielding supernatural powers should stress their Intellect stat.

(Intellect could be thought of as Intellect/Personality because it governs both intelligence and charisma.)

\subsection{Pool, Edge and Effort}\index{Core Rules!Introduction!Character Stats!Pool, Edge and Effort}

Each of the three stats has two components: Pool and Edge. Your Pool represents your raw, innate ability, and your Edge represents knowing how to use what you have. A third element ties into this concept: Effort. When your character really needs to accomplish a task, you apply Effort.

(Your stat Pools, as well as your Effort and Edge, are determined by the character type, descriptor, and focus that you choose. Within those guidelines, however, you have a lot of flexibility in how you develop your character.)

\paragraph{Pool}\index{Core Rules!Introduction!Character Stats!Pool, Edge and Effort!Pool}

Your Pool is the most basic measurement of a stat. Comparing the Pools of two creatures will give you a general sense of which creature is superior in that stat. For example, a character who has a Might Pool of 16 is stronger (in a basic sense) than a character who has a Might Pool of 12. Most characters start with a Pool of 9 to 12 in most stats—that’s the average range.

When your character is injured, sickened, or attacked, you temporarily lose points from one of your stat Pools. The nature of the attack determines which Pool loses points. For example, physical damage from a sword reduces your Might Pool, a poison that makes you clumsy reduces your Speed Pool, and a psionic blast reduces your Intellect Pool. You can also spend points from one of your stat Pools to decrease a task’s difficulty (see Effort, below). You can rest to recover lost points from a stat Pool, and some special abilities or cyphers might allow you to recover lost points quickly.

\paragraph{Edge}\index{Core Rules!Introduction!Character Stats!Pool, Edge and Effort!Edge}

Although your Pool is the basic measurement of a stat, your Edge is also important. When something requires you to spend points from a stat Pool, your Edge for that stat reduces the cost. It also reduces the cost of applying Effort to a roll.

For example, let’s say you have a mental blast ability, and activating it costs 1 point from your Intellect Pool. Subtract your Intellect Edge from the activation cost, and the result is how many points you must spend to use the mental blast. If using your Edge reduces the cost to 0, you can use the ability for free.

Your Edge can be different for each stat. For example, you could have a Might Edge of 1, a Speed Edge of 1, and an Intellect Edge of 0. You’ll always have an Edge of at least 1 in one stat. Your Edge for a stat reduces the cost of spending points from that stat Pool, but not from other Pools. Your Might Edge reduces the cost of spending points from your Might Pool, but it doesn’t affect your Speed Pool or Intellect Pool. Once a stat’s Edge reaches 3, you can apply one level of Effort for free.

A character who has a low Might Pool but a high Might Edge has the potential to perform Might actions consistently better than a character who has a Might Edge of 0. The high Edge will let them reduce the cost of spending points from the Pool, which means they’ll have more points available to spend on applying Effort.

\paragraph{Effort}\index{Core Rules!Introduction!Character Stats!Pool, Edge and Effort!Effort}

When your character really needs to accomplish a task, you can apply Effort. For a beginning character, applying Effort requires spending 3 points from the stat Pool appropriate to the action. Thus, if your character tries to dodge an attack (a Speed roll) and wants to increase the chance for success, you can apply Effort by spending 3 points from your Speed Pool. Effort eases the task by one step. This is called applying one level of Effort.

You don’t have to apply Effort if you don’t want to. If you choose to apply Effort to a task, you must do it before you attempt the roll—you can’t roll first and then decide to apply Effort if you rolled poorly.

Applying more Effort can lower a task’s difficulty further: each additional level of Effort eases the task by another step. Applying one level of Effort eases the task by one step, applying two levels eases the task by two steps, and so on. However, each level of Effort after the first costs only 2 points from the stat Pool instead of 3. So applying two levels of Effort costs 5 points (3 for the first level plus 2 for the second level), applying three levels costs 7 points (3 plus 2 plus 2), and so on.

Every character has an Effort score, which indicates the maximum number of levels of Effort that can be applied to a roll. A beginning (first-tier) character has an Effort of 1, meaning you can apply only one level of Effort to a roll. A more experienced character has a higher Effort score and can apply more levels of Effort to a roll. For example, a character who has an Effort of 3 can apply up to three levels of Effort to reduce a task’s difficulty.

When you apply Effort, subtract your relevant Edge from the total cost of applying Effort. For example, let’s say you need to make a Speed roll. To increase your chance for success, you decide to apply one level of Effort, which will ease the task. Normally, that would cost 3 points from your Speed Pool. However, you have a Speed Edge of 2, so you subtract that from the cost. Thus, applying Effort to the roll costs only 1 point from your Speed Pool.

What if you applied two levels of Effort to the Speed roll instead of just one? That would ease the task by two steps. Normally, it would cost 5 points from your Speed Pool, but after subtracting your Speed Edge of 2, it costs only 3 points.

Once a stat’s Edge reaches 3, you can apply one level of Effort for free. For example, if you have a Speed Edge of 3 and you apply one level of Effort to a Speed roll, it costs you 0 points from your Speed Pool. (Normally, applying one level of Effort would cost 3 points, but you subtract your Speed Edge from that cost, reducing it to 0.)

Skills and other advantages also ease a task, and you can use them in conjunction with Effort. In addition, your character might have special abilities or equipment that allow you to apply Effort to accomplish a special effect, such as knocking down a foe with an attack or affecting multiple targets with a power that normally affects only one.

(When applying Effort to melee attacks, you have the option of spending points from either your Might Pool or your Speed Pool. When making ranged attacks, you may spend points only from your Speed Pool. This reflects that with melee you sometimes use brute force and sometimes use finesse, but with ranged attacks, it’s always about careful targeting.)

\paragraph{Effort and Damage}\index{Core Rules!Introduction!Character Stats!Pool, Edge and Effort!Effort and Damage}

nstead of applying Effort to ease your attack, you can apply Effort to increase the amount of damage you inflict with an attack. For each level of Effort you apply in this way, you inflict 3 additional points of damage. This works for any kind of attack that inflicts damage, whether a sword, a crossbow, a mind blast, or something else.

When using Effort to increase the damage of an area attack, such as the explosion created by an Adept’s Concussion ability, you inflict 2 additional points of damage instead of 3 points. However, the additional points are dealt to all targets in the area. Further, even if one or more of the targets resist the attack, they still take 1 point of damage.

\paragraph{Multiple Uses of Effort and Edge}\index{Core Rules!Introduction!Character Stats!Pool, Edge and Effort!Multiple Uses of Effort and Edge}

f your Effort is 2 or higher, you can apply Effort to multiple aspects of a single action. For example, if you make an attack, you can apply Effort to your attack roll and apply Effort to increase the damage.

The total amount of Effort you apply can’t be higher than your Effort score. For example, if your Effort is 2, you can apply up to two levels of Effort. You could apply one level to an attack roll and one level to its damage, two levels to the attack and no levels to the damage, or no levels to the attack and two levels to the damage.

You can use Edge for a particular stat only once per action. For example, if you apply Effort to a Might attack roll and to your damage, you can use your Might Edge to reduce the cost of one of those uses of Effort, not both. If you spend 1 Intellect point to activate your mind blast and one level of Effort to ease the attack roll, you can use your Intellect Edge to reduce the cost of one of those things, not both.

\subsubsection{Character Stats}\index{Core Rules!Introduction!Character Stats!Stats Examples}

A beginning character is fighting a giant rat. The PC stabs their spear at the rat, which is a level 2 creature and thus has a target number of 6. The character stands atop a boulder and strikes downward at the beast, and the GM rules that this helpful tactic is an asset that eases the attack by one step (to difficulty 1). That lowers the target number to 3. Attacking with a spear is a Might action; the character has a Might Pool of 11 and a Might Edge of 0. Before making the roll, they decide to apply a level of Effort to ease the attack. That costs 3 points from their Might Pool, reducing the Pool to 8. But the points are well spent. Applying the Effort lowers the difficulty from 1 to 0, so no roll is needed—the attack automatically succeeds.

Another character is attempting to convince a guard to let them into a private office to speak to an influential noble. The GM rules that this is an Intellect action. The character is third tier and has an Effort of 3, an Intellect Pool of 13, and an Intellect Edge of 1. Before making the roll, they must decide whether to apply Effort. They can choose to apply one, two, or three levels of Effort, or apply none at all. This action is important to them, so they decide to apply two levels of Effort, easing the task by two steps. Thanks to their Intellect Edge, applying the Effort costs only 4 points from their Intellect Pool (3 points for the first level of Effort plus 2 points for the second level minus 1 point for their Edge). Spending those points reduces their Intellect Pool to 9. The GM decides that convincing the guard is a difficulty 3 (demanding) task with a target number of 9; applying two levels of Effort reduces the difficulty to 1 (simple) and the target number to 3. The player rolls a d20 and gets an 8. Because this result is at least equal to the target number of the task, they succeed. However, if they had not applied some Effort, they would have failed because their roll (8) would have been less than the task’s original target number (9).

\subsection{Character Tiers}\index{Core Rules!Introduction!Character Tiers}

Every character starts the game at the first tier. Tier is a measurement of power, toughness, and ability. Characters can advance up to the sixth tier. As your character advances to higher tiers, you gain more abilities, increase your Effort, and can improve a stat’s Edge or increase a stat. Generally speaking, even first-tier characters are already quite capable. It’s safe to assume that they’ve already got some experience under their belt. This is not a “zero to hero” progression, but rather an instance of competent people refining and honing their capabilities and knowledge. Advancing to higher tiers is not really the goal of Cypher System characters, but rather a representation of how characters progress in a story.

To progress to the next tier, characters earn experience points (XP) by pursuing character arcs, going on adventures, and discovering new things—the system is about both discovery and exploration, as well as achieving personal goals. Experience points have many uses, and one use is to purchase character benefits. After your character purchases four character benefits, they advance to the next tier. Each benefit costs 4 XP, and you can purchase them in any order, but you must purchase one of each kind of benefit (and then advance to the next tier) before you can purchase the same benefit again. The four character benefits are as follows.

\textbf{Increasing Capabilities}: You gain 4 points to add to your stat Pools. You can allocate the points among the Pools however you wish.

\textbf{Moving Toward Perfection}: You add 1 to your Might Edge, your Speed Edge, or your Intellect Edge (your choice).

\textbf{Extra Effort}: Your Effort score increases by 1.

\textbf{Skills}: You become trained in one skill of your choice, other than attacks or defense. As described in Rules of the Game, a character trained in a skill treats the difficulty of a related task as one step lower than normal. The skill you choose for this benefit can be anything you wish, such as climbing, jumping, persuading, or sneaking. You can also choose to be knowledgeable in a certain area of lore, such as history or geology. You can even choose a skill based on your character’s special abilities. For example, if your character can make an Intellect roll to blast an enemy with mental force, you can become trained in using that ability, easing the task of using it. If you choose a skill that you are already trained in, you become specialized in that skill, easing related tasks by two steps instead of one.

(Skills are a broad category of things your character can learn and accomplish. For a list of sample skills, see below.)

\textbf{Other Options}: Players can also spend 4 XP to purchase other special options in lieu of gaining a new skill. Selecting any of these options counts as the skill benefit necessary to advance to the next tier. The special options are as follows:

\begin{itemize}
\item Reduce the cost for wearing armor. This option lowers the Speed cost for wearing armor by 1. 
\item Add 2 to your recovery rolls.
\item Select a new type-based ability from your tier or a lower tier.
\end{itemize}

\subsection{Character Descriptor, Type, and Focus}\index{Core Rules!Introduction!Character Descriptor, Type and Focus}

To create your character, you build a simple statement that describes them. The statement takes this form: “I am a [fill in an adjective here] [fill in a noun here] who [fill in a verb here].”

Thus: “I am an adjective noun who verbs.” For example, you might say, “I am a Rugged Warrior who Controls Beasts” or “I am a Charming Explorer who Focuses Mind Over Matter.”

In this sentence, the adjective is called your descriptor.

The noun is your character type.

The verb is called your focus.

Even though character type is in the middle of the sentence, that’s where we’ll start this discussion. (Just as in a sentence, the noun provides the foundation.)
Your character type is the core of your character. In some roleplaying games, it might be called your character class. Your type helps determine your character’s place in the world and relationship with other people in the setting. It’s the noun of the sentence “I am an adjective noun who verbs.”

You can choose from four character types: Warriors, Adepts, Explorers, and Speakers.

Your descriptor defines your character—it colors everything you do. Your descriptor places your character in the situation (the first adventure, which starts the campaign) and helps provide motivation. It’s the adjective of the sentence “I am an adjective noun who verbs.”

Unless your GM says otherwise, you can choose from any of the character descriptors.

Focus is what your character does best. Focus gives your character specificity and provides interesting new abilities that might come in handy. Your focus also helps you understand how you relate with the other player characters in your group. It’s the verb of the sentence “I am an adjective noun who verbs.”
There are many character foci. The ones you choose from will probably depend on the setting and genre of your game.

(You can use the Flavors chapter to slightly modify character types to customize them for different genres.)

\subsection{Special Abilities}\index{Core Rules!Introduction!Special Abilities}

Character types and foci grant PCs special abilities at each new tier. Using these abilities usually costs points from your stat Pools; the cost is listed in parentheses after the ability name. Your Edge in the appropriate stat can reduce the cost of the ability, but remember that you can apply Edge only once per action. For example, let’s say an Adept with an Intellect Edge of 2 wants to use their Onslaught ability to create a bolt of force, which costs 1 Intellect point. They also want to increase the damage from the attack by using a level of Effort, which costs 3 Intellect points. The total cost for their action is 2 points from their Intellect Pool (1 point for the bolt of force, plus 3 points for using Effort, minus 2 points from their Edge).

Sometimes the point cost for an ability has a + sign after the number. For example, the cost might be given as “2+ Intellect points.” That means you can spend more points or more levels of Effort to improve the ability further, as explained in the ability description.

Many special abilities grant a character the option to perform an action that they couldn’t normally do, such as projecting rays of cold or attacking multiple foes at once. Using one of these abilities is an action unto itself, and the end of the ability’s description says “Action” to remind you. It also might provide more information about when or how you perform the action.

Some special abilities allow you to perform a familiar action—one that you can already do—in a different way. For example, an ability might let you wear heavy armor, reduce the difficulty of Speed defense rolls, or add 2 points of fire damage to your weapon damage. These abilities are called enablers. Using one of these abilities is not considered an action. Enablers either function constantly (such as being able to wear heavy armor, which isn’t an action) or happen as part of another action (such as adding fire damage to your weapon damage, which happens as part of your attack action). If a special ability is an enabler, the end of the ability’s description says “Enabler” to remind you.

Some abilities specify a duration, but you can always end one of your own abilities anytime you wish.

(Because the Cypher System covers so many genres, not all of the descriptors, types, and foci might be available for players. The GM will decide what’s available in their particular game and whether anything is modified, and they’ll let the players know.)

\subsection{Skills}\index{Core Rules!Introduction!Skills}

Sometimes your character gains training in a specific skill or task. For example, your focus might mean that you’re trained in sneaking, in climbing and jumping, or in social interactions. Other times, your character can choose a skill to become trained in, and you can pick a skill that relates to any task you think you might face.

The Cypher System has no definitive list of skills. However, the following list offers ideas:

\begin{itemize}
\item Astronomy
\item Balancing
\item Biology
\item Botany
\item Carrying
\item Climbing
\item Computers
\item Deceiving
\item Disguise
\item Escaping
\item Geography
\item Geology
\item Healing
\item History
\item Identifying
\item Initiative
\item Intimidation
\item Jumping
\item Leatherworking
\item Lockpicking
\item Machinery
\item Metalworking
\item Perception
\item Persuasion
\item Philosophy
\item Physics
\item Pickpocketing
\item Piloting
\item Repairing
\item Riding
\item Smashing
\item Sneaking
\item Stealth
\item Swimming
\item Vehicle driving
\item Woodworking 
\end{itemize}

You could choose a skill that incorporates more than one of these areas (interacting might include deceiving, intimidation, and persuasion) or that is a more specific version of one (hiding might be sneaking when you’re not moving). You could also make up more general professional skills, such as baker, sailor, or lumberjack. If you want to choose a skill that’s not on this list, it’s probably best to run it past the GM first, but in general, the most important thing is to choose skills that are appropriate to your character.

Remember that if you gain a skill that you’re already trained in, you become specialized in that skill. Because skill descriptions can be nebulous, determining whether you’re trained or specialized might take some thinking. For example, if you’re trained in lying and later gain an ability that grants you skill with all social interactions, you become specialized in lying and trained in all other types of interactions. Being trained three times in a skill is no better than being trained twice (in other words, specialized is as good as it gets).

Only skills gained through character type abilities or other rare instances allow you to become skilled with attack or defense tasks.

If you gain a special ability through your type, your focus, or some other aspect of your character, you can choose it in place of a skill and become trained or specialized in that ability. For example, if you have a mind blast, when it’s time to choose a skill to be trained in, you can select your mind blast as your skill. That would ease the attack every time you used it. Each ability you have counts as a separate skill for this purpose. You can’t select “all mind powers” or “all spells” as one skill and become trained or specialized in such a broad category.

In most campaigns, fluency in a language is considered a skill. So if you want to speak French, that’s the same as being trained in biology or swimming.

\subsection{Type}\index{Core Rules!Introduction!Type}

Character type is the core of your character. Your type helps determine your character’s place in the world and relationship with other people in the setting. It’s the noun of the sentence “I am an adjective noun who verbs.”

(In some roleplaying games, your character type might be called your character class.)

You can choose from four character types: Warrior, Adept, Explorer, and Speaker. However, you may not want to use these generic names for them. This chapter offers a few more specific names for each type that might be more appropriate to various genres. You’ll find that names like “Warrior” or “Explorer” don’t always feel right, particularly in games set in modern times. As always, you’re free to do as you wish. (Your type is who your character is. You should use whatever name you want for your type, as long as it fits both your character and the setting.)

Since the type is the basis upon which your whole character is built, it’s important to consider how the type relates to the chosen setting. To help with this, types are actually general archetypes. A Warrior, for example, might be anyone from a knight in shining armor to a cop on the streets to a grizzled cybernetic veteran of a thousand futuristic wars. 

To further massage the four types for better use in various settings, different methods called flavors are presented in Flavors to help slightly tailor the types toward fantasy, science fiction, or other genres (or to address different character concepts).

Finally, more fundamental options for further customization are provided at the end of this chapter.

\subsection{Player Intrusion}\index{Core Rules!Introduction!Player Intrusion}

A player intrusion is the player choosing to alter something in the campaign, making things easier for a player character. Conceptually, it is the reverse of a GM intrusion: instead of the GM giving the player XP and introducing an unexpected complication for a character, the player spends 1 XP and presents a solution to a problem or complication. What a player intrusion can do usually introduces a change to the world or current circumstances rather than directly changing the character. For instance, an intrusion indicating that the cypher just used still has an additional use would be appropriate, but an intrusion that heals the character would not. If a player has no XP to spend, they can’t use a player intrusion.

A few player intrusion examples are provided under each type. That said, not every player intrusion listed there is appropriate for all situations. The GM may allow players to come up with other player intrusion suggestions, but the GM is the final arbiter of whether the suggested intrusion is appropriate for the character’s type and suitable for the situation. If the GM refuses the intrusion, the player doesn’t spend the 1 XP, and the intrusion doesn’t occur.

Using an intrusion does not require a character to use an action to trigger it. A player intrusion just happens.

(Player intrusions should be limited to no more than one per player per session.)

\subsection{Defense Task}\index{Core Rules!Introduction!Defense Task}

Defense tasks are when a player makes a roll to keep something undesirable from happening to their PC. The type of defense task matters when using Effort.

\textbf{Might defense}: Used for resisting poison, disease, and anything else that can be overcome with strength and health.

\textbf{Speed defense}: Used for dodging attacks and escaping danger. This is by far the most commonly used defense task.

\textbf{Intellect defense}: Used for fending off mental attacks or anything that might affect or influence one’s mind.

\end{multicols}
