\chapter{Fantasy Character Options}\index{Fantasy Character Options}

In some cases, the ideas here require minor changes to the flavor described in the character options; you should work with your GM to make sure these changes are suitable for the campaign. Most of the foci in this section appear in the Cypher System; foci with an asterisk (*) are found later in this document. Some of these options recommend swapping out a type ability for an ability from one of the character flavors such as combat, magic, or stealth. 

Alchemist: In the sense that an alchemist is someone who makes magical items or similar types of things, Adept and Explorer are appropriate type choices for academic alchemists. For a general sort of alchemist who makes potions of magical effects, choose the Masters Spells focus (instead of spells, you learn potions). For one who transforms into a powerful and dangerous creature, choose Howls at the Moon. For one who loves throwing bombs, choose Bears a Halo of Fire. For a healer, choose Works Miracles.

Assassin/Spy: Explorer and Warrior are good type choices for an assassin character. Appropriate foci are Masters Weaponry, Moves Like a Cat, Murders, and Works the Back Alleys. 

Barbarian: A barbarian character is probably a Warrior or (to focus a little more on skills than combat) an Explorer. Good foci to choose from are Lives in the Wilderness, Masters Weaponry, Needs No Weapon, Never Says Die, Performs Feats of Strength, and Rages. 

Bard: Bards in fantasy fiction and games are troubadours, minstrels, and storytellers, perhaps with a supernatural element. Bards are usually Explorers or 
Speakers. Appropriate foci are Entertains, Helps Their Friends, Infiltrates, and Masters Spells. 

Cleric or Priest: Academic clerics are usually Adepts or Speakers, but martial clerics are often Warriors (perhaps with magic flavor). For a typical cleric with a versatile set of abilities, choose the Channels Divine Blessings focus. 

Cleric (death): Consorts With the Dead, Shepherds Spirits 

Cleric (knowledge): Learns Quickly, Sees Beyond, Would Rather Be Reading 

Cleric (life): Defends the Weak, Shepherds the Community, Works Miracles

Cleric (light): Blazes With Radiance, Channels Divine Blessings 

Cleric (storm): Rides the Lightning, Thunders 

Cleric (trickery): Takes Animal Shape* (also see options for rogues)

Cleric (war): Masters Weaponry (also see options for fighters)

Druid: As a very specific sort of nature priest, a druid character is usually an Adept or Explorer (in either case probably using the magic flavor). A typical druid probably has Channels Divine Blessings or Lives in the Wilderness as a focus, but for more specific options, see the following foci: 

Druid (animal companion): Controls Beasts, Masters the Swarm 

Druid (elemental): Abides in Stone, Bears a Halo of Fire, Moves Like the Wind, Rides the Lightning, Wears a Sheen of Ice 

Druid (nature affinity): Speaks for the Land 

Druid (transformation): Abides in Stone, Takes Animal Shape*, Walks the Wild Woods*

Fighter: Fighters almost always have the Warrior type, but some are Explorers. A typical fighter probably has a direct focus like Masters Weaponry or Wields an Enchanted Weapon*. For additional options based on choosing a specific fighting role, see the following: 

Fighter (guardian): Brandishes an Exotic Shield, Defends the Gate, Masters Defense, Never Says Die, Stands Like a Bastion. 

Fighter (melee): Fights Dirty, Fights With Panache, Looks For Trouble, Needs No Weapon, Wields Two Weapons at Once 

Fighter (ranged): Is Licensed to Carry, Throws With Deadly Accuracy
Gunslinger: A gunslinger is probably a Warrior or Explorer, but some are Speakers with combat flavor. Appropriate foci are Is Licensed to Carry, Masters Weaponry, 
Sailed Beneath the Jolly Roger, and Wields an Enchanted Weapon*. 

Inquisitor: Inquisitors are usually Explorers, Speakers, or Warriors, depending on whether their inclinations are for having many skills, being good at interacting with people, or combat. Appropriate foci are Infiltrates, Metes Out Justice, and Operates Undercover. 

Merchant: An Explorer with a focus dealing with social interactions, like Entertains or Leads, would make a good merchant character, but the more obvious choice would be a Speaker.

Monk or Martial Artist: As masters of unarmed combat, monks are usually Warriors or Explorers (perhaps with a combat flavor). Appropriate foci are Fights With Panache, Needs No Weapon, and Throws With Deadly Accuracy. 

Paladin/Holy Knight/Paragon: As holy warriors who mix martial prowess and magic, paladins are usually Warriors or Explorers (in either case, perhaps modified with the magic flavor). Good foci for this type of character include Defends the Gate, Defends the Weak, Metes Out Justice, Slays Monsters, and Wields an Enchanted Weapon*.

Ranger: Rangers mix combat and skills, and therefore are usually Explorers (perhaps with combat flavor) or Warriors (perhaps with skills and knowledge flavor). Appropriate foci for a ranger are Controls Beasts, Hunts, Lives in the Wilderness, Slays Monsters, Throws With Deadly Accuracy, and Wields Two Weapons at Once. 

Rogue or Thief: Most rogue-type characters are Explorers, but an interaction-focused rogue could easily be a Speaker (perhaps with stealth flavor). Good foci for rogues are Explores Dark Places, Fights Dirty, Hunts, Infiltrates, Is Wanted by the Law, Moves Like a Cat, Sailed Beneath the Jolly Roger, and Works the Back Alleys. 

Sorcerer: Sorcerers, for our purpose here, are mages who have inherent magical abilities (as opposed to wizards, who study long and hard to get their spells). Most sorcerers are Adepts, but some are Explorers or Speakers. The Masters Spells focus gives a typical sorcerer an effective set of abilities, and most foci choices provide a themed set of spells. For sorcerers of various magical bloodlines, see the following: 

Sorcerer (angel): Blazes With Radiance, Channels Divine Blessings, Keeps a Magic Ally 

Sorcerer (destiny): Descends From Nobility, Was Foretold

Sorcerer (dragon): Bears a Halo of Fire, Rides the Lightning, Wears a Sheen of Ice

Sorcerer (elemental): Abides in Stone, Bears a Halo of Fire, Employs Magnetism, Moves Like the Wind, Rides the Lightning, Wears a Sheen of Ice 

Sorcerer (fey): Takes Animal Shape* 

Sorcerer (fiend): Bears a Halo of Fire, Keeps a Magic Ally 

Sorcerer (undead): Consorts With the Dead, Shepherds Spirits

Trickster or Con Artist: These clever folks are typically Speakers, although they could be Adepts if they are very magical (or Explorers if they aren’t magical at all). Foci choices include Fights Dirty, Works the Back Alleys, or Entertains. 

War-wizard: For those unusual characters who use a mix of weapon attacks and spells, play a Warrior with magic flavor or an Expert with combat or magic flavor. Appropriate foci include Fights With Panache, Masters Weaponry, and Wields an Enchanted Weapon*. 

Warlock or Witch: For the purposes of this list, warlocks and witches are mages who gain magical power from pacts they make with otherworldly entities. Most warlocks are Adepts, but Explorers and Speakers (perhaps with magic flavor) can be interesting options. Fun foci for a warlock include Dances With Dark Matter, Keeps a Magic Ally, Masters the Swarm, Separates Mind From Body, and Was Foretold, but (depending on the patron and pact) most sorcerer and wizard foci work just as well. 

Wild Mage: Those who use chaotic magic are usually Adepts, but a dabbler might be an Explorer or Speaker with the magic flavor. The best focus that suits this theme is Uses Wild Magic*.

Wizard: For the purposes of this list, wizards study magical lore at length to learn the ways of spellcasting (as opposed to sorcerers, warlocks, and so on). Wizards are usually Adepts, but a person-oriented wizard might be a Speaker (perhaps with the magic flavor). For a generalist wizard who has a variety of spells, choose the Masters Spells focus. For more specific kinds of wizards, see the following: 

Wizard (abjurer): Absorbs Energy, Focuses Mind Over Matter, Wears a Sheen of Ice 

Wizard (conjurer or summoner): Controls Beasts, Keeps a Magic Ally 

Wizard (diviner): Learns Quickly, Sees Beyond, Separates Mind From Body, Solves Mysteries 

Wizard (enchanter): Commands Mental Powers, Leads 

Wizard (evoker): Bears a Halo of Fire, Blazes With Radiance, Rides the Lightning, Thunders, Wears a Sheen of Ice 

Wizard (illusionist): Awakens Dreams, Crafts Illusions 

Wizard (necromancer): Consorts With the Dead, Shepherds Spirits 

Wizard (transmuter): Controls Gravity, Focuses Mind Over Matter, Takes Animal Shape*

\section{Prepared vs. Spontaneous Spellcasting}\index{Fantasy Character Options!Prepared vs. Spontaneous Spellcasting}

Magical characters get their abilities (which might be spells, rituals, or something else) from their type and focus, and they can use these abilities as they see fit as long as they spend the required Pool points. This technically makes them more like spontaneous casters. If you’d like to play something more like a prepared-caster wizard with a large selection of abilities that you narrow down each day, consider a spellcasting-oriented focus like Channels Divine Blessings, Masters Spells, or Speaks for the Land, and consider augmenting it with the optional spellcasting rule.

\chapter{Further Customization}\index{Further Customisation}
The rules in this section are more advanced and always involve the GM. They can be used by the GM to tailor a type to better fit the genre or setting, or by a player and a GM to tweak a character to fit a concept.

\section{Modifying Type Aspects}\index{Further Customisation!Modifying Type Aspects}

The following aspects of the four character types can be modified at character creation. Other abilities should not be changed.

\textbf{Stat Pools}: Each character type has a starting stat Pool value. A player can exchange points between their Pools on a one-for-one basis. For example, they can trade 2 points of Might for 2 points of Speed. However, no starting stat Pool should be higher than 20.

\textbf{Edge}: A player can start with an Edge of 1 in whichever stat they wish.

\textbf{Cypher Use}: If a character gives up the ability to bear one cypher, they gain an additional skill of their choice.

\textbf{Weapons}: Some types have static first-tier abilities that let them use light, medium, and/or heavy weapons without a penalty. Warriors can use all weapons, Explorers can use light and medium weapons, and Adepts and Speakers can use light weapons. Any one of these weapon abilities can be sacrificed to gain training in a different skill of the player’s choice.

\section{Drawbacks and Penalties}\index{Further Customisation!Drawbacks and Penalties}

In addition to other customization options, a player can choose to take drawbacks or penalties to gain further advantages.

\textbf{Weakness}: A weakness is, essentially, the opposite of Edge. If you have a weakness of 1 in Speed, all Speed actions that require you to spend points cost 1 additional point from your Pool. At any time, a player can give their character a weakness in one stat and, in exchange, gain +1 to their Edge in one of the other two stats. So a PC can take a weakness of 1 in Speed to gain +1 to their Might Edge.
Normally, you can have a weakness only in a stat in which you have an Edge of 0. Further, you can’t have more than one weakness, and you can’t have a weakness greater than 1 unless the additional weakness comes from another source (such as a disease or disability arising from actions or conditions in the game).

\textbf{Inabilities}: Inabilities are like negative skills. They make one type of task harder by hindering it. If a character chooses to take an inability, they gain a skill of their choice. Normally, a character can have only one inability unless the additional inability comes from another source (such as a descriptor or a disease or disability arising from actions or conditions in the game).

\section{Flavors}\index{Further Customisation!Flavors}

Flavors are groups of special abilities the GM and players can use to alter a character type to make it more to their liking or more appropriate to the genre or setting. For example, if a player wants to create a magic-using thief character, she could play an Adept with stealth flavoring. In a science fiction setting, a Warrior might also have knowledge of machinery, so the character could be flavored with technology.
At a given tier, abilities from a flavor are traded one for one with standard abilities from a type. So to add the Danger Sense stealth flavor ability to a Warrior, something else—perhaps Bash—must be sacrificed. Now that character can choose Danger Sense as they would any other first-tier warrior ability, but they can never choose Bash.
The GM should always be involved in flavoring a type. For example, they might know that for their science fiction game, they want a type called a “Glam,” which is a Speaker flavored with certain technology abilities—specifically those that make the character a flamboyant starship pilot. Thus, they exchange the first-tier abilities Spin Identity and Inspire Aggression for the technology flavor abilities Datajack and Tech Skills so the character can plug into the ship directly and can take piloting and computers as skills.
In the end, flavor is mostly a tool for the GM to easily create campaign-specific types by making a few slight alterations to the four base types. Although players may wish to use flavors to get the characters they want, remember that they can also shape their PCs with descriptors and foci very nicely.
The flavors available are stealth, technology, magic, combat, and skills and knowledge. 
The full description for each listed ability can be found in the Abilities chapter, which also contains descriptions for type and focus abilities in a single vast catalog.
STEALTH FLAVOR
Characters with the stealth flavor are good at sneaking around, infiltrating places they don’t belong, and deceiving others. They use these abilities in a variety of ways, including combat. An Explorer with stealth flavor might be a thief, while a Warrior with stealth flavor might be an assassin. An Explorer with stealth flavor in a superhero setting might be a crimefighter who stalks the streets at night.
FIRST-TIER STEALTH ABILITIES
Danger Sense
Goad
Legerdemain
Opportunist
Stealth Skills
SECOND-TIER STEALTH ABILITIES
Contortionist
Find an Opening
Get Away
Sense Ambush
Surprise Attack
THIRD-TIER STEALTH ABILITIES
Evanesce
From the Shadows
Gambler
Inner Defense
Misdirect
Run and Fight
Seize the Moment
FOURTH-TIER STEALTH ABILITIES
Ambusher
Debilitating Strike
Outwit
Preternatural Senses
Tumbling Moves
FIFTH-TIER STEALTH ABILITIES
Assassin Strike
Mask
Return to Sender
Uncanny Luck
SIXTH-TIER STEALTH ABILITIES
Exploit Advantage
Spring Away
Thief’s Luck
Twist of Fate
TECHNOLOGY FLAVOR
Characters with a flavor of technology typically are from science fiction or at least modern-day
settings (although anything is possible). They excel at using, dealing with, and building machines. An Explorer with technology flavor might be a starship pilot, and a Speaker flavored with technology could be a techno-priest.
Some of the less computer-oriented abilities might be appropriate for a steampunk character, while a modern-day character could use some of the abilities that don’t involve starships or ultratech.
FIRST-TIER TECHNOLOGY ABILITIES
Datajack
Hacker
Machine Interface
Scramble Machine
Tech Skills
Tinker
SECOND-TIER TECHNOLOGY ABILITIES
Distant Interface
Machine Efficiency
Overload Machine
Serv-0
Serv-0 Defender
Serv-0 Repair
Tool Mastery
THIRD-TIER TECHNOLOGY ABILITIES
Mechanical Telepathy
Serv-0 Scanner
Ship Footing
Shipspeak
Spray
FOURTH-TIER TECHNOLOGY ABILITIES
Machine Bond
Robot Fighter
Serv-0 Aim
Serv-0 Brawler
Serv-0 Spy
FIFTH-TIER TECHNOLOGY ABILITIES
Control Machine
Jury-Rig
Machine Companion 
SIXTH-TIER TECHNOLOGY ABILITIES
Information Gathering
Master Machine
MAGIC FLAVOR
You know a little about magic. You might not be a wizard, but you know the basics—how it works, and how to accomplish a few wondrous things. Of course, in your setting, “magic” might actually mean psychic powers, mutant abilities, weird alien tech, or anything else that produces interesting and useful effects. 
An Explorer flavored with magic might be a wizard-hunter, and a Speaker with magical flavor might be a sorcerer-bard. Although an Adept flavored with magic is still an Adept, you might find that swapping some of the type’s basic abilities with those given here tailors the character in desirable ways.
FIRST-TIER MAGIC ABILITIES
Blessing of the Gods
Closed Mind
Entangling Force
Hedge Magic
Magic Training
Mental Link
Premonition
SECOND-TIER MAGIC ABILITIES
Concussive Blast
Fetch
Force Field 
Lock
Repair Flesh
THIRD-TIER MAGIC ABILITIES
Distance Viewing
Fire Bloom
Fling
Force at Distance
Summon Giant Spider
FOURTH-TIER MAGIC ABILITIES
Elemental Protection
Ignition
Pry Open
FIFTH-TIER MAGIC ABILITIES
Create
Divine Intervention
Dragon’s Maw
Fast Travel
True Senses
SIXTH-TIER MAGIC ABILITIES
Relocate
Summon Demon
Traverse the Worlds
Word of Death
COMBAT FLAVOR
Combat flavor makes a character more martial. A Speaker with combat flavor in a fantasy setting would be a battle bard. An Explorer with combat flavor in a historical game might be a pirate. An Adept flavored with combat in a science fiction setting could be a veteran of a thousand psychic wars.
FIRST-TIER COMBAT ABILITIES
Danger Sense
Practiced in Armor
Practiced With Medium Weapons
SECOND-TIER COMBAT ABILITIES
Bloodlust
Combat Prowess
Trained Without Armor
THIRD-TIER COMBAT ABILITIES
Practiced With All Weapons
Skill With Attacks
Skill With Defense
Successive Attack
FOURTH-TIER COMBAT ABILITIES
Capable Warrior
Deadly Aim
Fury
Misdirect
Spray
FIFTH-TIER COMBAT ABILITIES
Experienced Defender
Hard Target
Parry
SIXTH-TIER COMBAT ABILITIES
Greater Skill With Attacks
Mastery in Armor
Mastery With Defense
SKILLS AND KNOWLEDGE FLAVOR
This flavor is for characters in roles that call for more knowledge and more real-world application of talent. It’s less flashy and dramatic than supernatural powers or the ability to hack apart multiple foes, but sometimes expertise or
know-how is the real solution to a problem.
A Warrior flavored with skills and knowledge might be a military engineer. An Explorer flavored with skills and knowledge could be a field scientist. A Speaker with this flavor might be a teacher.
FIRST-TIER SKILLS AND KNOWLEDGE ABILITIES 
Interaction Skills
Investigative Skills 
Knowledge Skills
Physical Skills
Travel Skills
SECOND-TIER SKILLS AND KNOWLEDGE ABILITIES 
Extra Skill
Tool Mastery
Understanding
THIRD-TIER SKILLS ANDKNOWLEDGE ABILITIES 
Flex Skill
Improvise
FOURTH-TIER SKILLS AND KNOWLEDGE ABILITIES 
Multiple Skills
Quick Wits
Task Specialization
FIFTH-TIER SKILLS ANd KNOWLEDGE ABILITIES
Practiced With Medium Weapons
Read the Signs
SIXTH-TIER SKILLS AND KNOWLEDGE ABILITIES
Skill With Attacks
Skill With Defense
