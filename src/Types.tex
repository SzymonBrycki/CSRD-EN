\chapter{Types}\index{Types}

\section{Warrior}\index{Types!Warrior}

\textbf{Fantasy/Fairy tale}: Warrior, fighter, swordsman, knight, barbarian, soldier, myrmidon, valkyrie

\textbf{Modern/Horror/Romance}: police officer, soldier, watchman, detective, guard, brawler, tough, athlete

\textbf{Science fiction}: security officer, warrior, trooper, soldier, merc

\textbf{Superhero/Post-Apocalyptic}: hero, brick, bruiser

You’re a good ally to have in a fight. You know how to use weapons and defend yourself. Depending on the genre and setting in question, this might mean wielding a sword and shield in the gladiatorial arena, an AK-47 and a bandolier of grenades in a savage firefight, or a blaster rifle and powered armor when exploring an alien planet.

\textbf{Individual Role}: Warriors are physical, action-oriented people. They’re more likely to overcome a challenge using force than by other means, and they often take the most straightforward path toward their goals. 

\textbf{Group Role}: Warriors usually take and deal the most punishment in a dangerous situation. Often it falls on them to protect the other group members from threats. This sometimes means that warriors take on leadership roles as well, at least in combat and other times of danger.

\textbf{Societal Role}: Warriors aren’t always soldiers or mercenaries. Anyone who is ready for violence, or even potential violence, might be a Warrior in the general sense. This includes guards, watchmen, police officers, sailors, or people in other roles or professions who know how to defend themselves with skill. 

\textbf{Advanced Warriors}: As warriors advance, their skill in battle—whether defending themselves or dishing out damage—increases to impressive levels. At higher tiers, they can often take on groups of foes by themselves or stand toe to toe with anyone.

\subsection{Warrior Stat Pools}\index{Types!Warrior!Stat Pools}

\begin{table}[H]
\centering
\caption{Warrior Stat Pools}
\label{tab: Warrior Stat Pools}

\begin{tabularx}{\linewidth}{| X | X |}
% \begin{tabularx}{\linewidth}{| p{0.5\linewidth} | p{0.5\linewidth} |}
\hline
\textbf{Stat} & \textbf{Pool Starting Value} \\
\hline
% Tutaj wpisujesz dane do tabeli
Might & 10 \\ \hline
Speed & 10 \\ \hline
Intelect & 8  \\ \hline

\end{tabularx}

\end{table}

\raggedright

You get 6 additional points to divide among your stat Pools however you wish.

\subsection{Warrior Background Connection}\index{Types!Warrior!Background Connection}

Your type helps determine the connection you have to the setting. Roll a d20 or choose from the following list to determine a specific fact about your background that provides a connection to the rest of the world. You can also create your own fact.

\begin{table*}
\centering

\caption{Warrior Background Connection}
\label{tab:Warrior Background Connection}

\begin{tabularx}{\textwidth}{| p{0.05\textwidth} | p{0.9\textwidth} |}
\hline
\textbf{d20} & \textbf{Background} \\
\hline
% Tutaj wpisujesz dane do tabeli
1 & You were in the military and have friends who still serve. Your former commander remembers you well.\\ \hline
2 & You were the bodyguard of a wealthy woman who accused you of theft. You left her service in disgrace.  \\ \hline
3 & You were the bouncer in a local bar for a while, and the patrons there remember you.  \\ \hline
4 & You trained with a highly respected mentor. They regard you well, but they have many enemies.  \\ \hline
5 & You trained in an isolated monastery. The monks think of you as a brother, but you’re a stranger to all others.  \\ \hline
6 & You have no formal training. Your abilities come to you naturally (or unnaturally).  \\ \hline
7 & You spent time on the streets and were in prison for a while. \\ \hline
8 & You were conscripted into military service, but you deserted before long.  \\ \hline
9 & You served as a bodyguard to a powerful criminal who now owes you their life.  \\ \hline
10 & You worked as a police officer or constable of some kind. Everyone knows you, but their opinions of you vary.  \\ \hline
11 & Your older sibling is an infamous character who has been disgraced. \\ \hline
12 & You served as a guard for someone who traveled extensively. You know a smattering of people in many locations. \\ \hline
13 & Your best friend is a teacher or scholar. They are a great source of knowledge. \\ \hline
14 & You and a friend both smoke the same kind of rare, expensive tobacco. The two of you get together weekly to chat and smoke. \\ \hline
15 & Your uncle runs a theater in town. You know all the actors and watch all the shows for free. \\ \hline
16 & Your craftsman friend sometimes calls on you for help. However, they pay you well. \\ \hline
17 & Your mentor wrote a book on martial arts. Sometimes people seek you out to ask about its stranger passages. \\ \hline
18 & Someone you fought alongside in the military is now the mayor of a nearby town. \\ \hline
19 & You saved the lives of a family when their house burned down. They’re indebted to you, and their neighbors regard you as a hero. \\ \hline
20 & Your old trainer still expects you to come back and clean up after their classes; when you do, they occasionally share interesting rumors. \\ \hline

\end{tabularx}

\end{table*}

\subsection{Warrior Player Intrusions}\index{Types!Warrior!Player Intrusions}

You can spend 1 XP to use one of the following player intrusions, provided the situation is appropriate and the GM agrees.

\textbf{Perfect Setup}: You’re fighting at least three foes and each one is standing in exactly the right spot for you to use a move you trained in long ago, allowing you to attack all three as a single action. Make a separate attack roll for each foe. You remain limited by the amount of Effort you can apply on one action.

\textbf{Old Friend}: A comrade in arms from your past shows up unexpectedly and provides aid in whatever you’re doing. They are on a mission of their own and can’t stay longer than it takes to help out, chat for a while after, and perhaps share a quick meal.
Weapon Break: Your foe’s weapon has a weak spot. In the course of the combat, it quickly becomes damaged and moves two steps down the object damage track.

\subsection{First-tier Warrior}\index{Types!Warrior!1st Tier Warrior}

First-tier warriors have the following abilities:

\textbf{Effort}: Your Effort is 1.

\textbf{Physical Nature}: You have a Might Edge of 1 and a Speed Edge of 0, or you have a Might Edge of 0 and a Speed Edge of 1. Either way, you have an Intellect Edge of 0.

\textbf{Cypher Use}: You can bear two cyphers at a time.

\textbf{Weapons}: You become practiced with light, medium, and heavy weapons and suffer no penalty when using any kind of weapon. Enabler.

\textbf{Starting Equipment}: Appropriate clothing and two weapons of your choice, plus one expensive item, two moderately priced items, and up to four inexpensive items.

\textbf{Special Abilities}: Choose four of the abilities listed below. You can’t choose the same ability more than once unless its description says otherwise. The full description for each listed ability can be found in Abilities, which also has descriptions for flavor and focus abilities in a single vast catalog.

\begin{itemize}
\item Bash
\item Combat Prowess
\item Control the Field
\item Improved Edge
\item No Need for Weapons
\item Overwatch
\item Physical Skills
\item Practiced in Armor
\item Quick Throw
\item Swipe
\item Trained Without Armor
\end{itemize}

\subsection{Second-tier Warrior}\index{Types!Warrior!2nd Tier Warrior}

Choose two of the abilities listed below (or from a lower tier) to add to your repertoire. In addition, you can replace one of your lower-tier abilities with a different one from a lower tier.

\begin{itemize}
\item Crushing Blow
\item Hemorrhage
\item Reload
\item Skill With Attacks
\item Skill With Defense
\item Successive Attack
\end{itemize}


\subsection{Third-tier Warrior}\index{Types!Warrior!3rd Tier Warrior}

Choose three of the abilities listed below (or from a lower tier) to add to your repertoire. In addition, you can replace one of your lower-tier abilities with a different one from a lower tier.

\begin{itemize}
\item Deadly Aim
\item Energy Resistance
\item Experienced in Armor
\item Expert Cypher Use
\item Fury
\item Lunge
\item Reaction
\item Seize the Moment
\item Slice
\item Spray
\item Trick Shot
\item Vigilance
\end{itemize}

\subsection{Fourth-tier Warrior}\index{Types!Warrior!4th Tier Warrior}

Choose two of the abilities listed below (or from a lower tier) to add to your repertoire. In addition, you can replace one of your lower-tier abilities with a different one from a lower tier.

\begin{itemize}
\item Amazing Effort
\item Capable Warrior
\item Experienced Defender
\item Feint
\item Increased Effects
\item Momentum
\item Pry Open
\item Snipe
\item Tough As Nails
\end{itemize}

\subsection{Fifth-tier Warrior}\index{Types!Warrior!5th Tier Warrior}

Choose three of the abilities listed below (or from a lower tier) to add to your repertoire. In addition, you can replace one of your lower-tier abilities with a different one from a lower tier.

\begin{itemize}
\item Adroit Cypher Use
\item Arc Spray
\item Improved Success
\item Jump Attack
\item Mastery in Armor
\item Mastery With Attacks
\item Mastery With Defense
\item Parry
\end{itemize}

(Remember that at higher tiers, you can choose special abilities from lower tiers. This is sometimes the best way to ensure that you have exactly the character you want. This is particularly true with abilities that grant skills, which can usually be taken multiple times.)

\subsection{Sixth-tier Warrior}\index{Types!Warrior!6th Tier Warrior}

Choose two of the abilities listed below (or from a lower tier) to add to your repertoire. In addition, you can replace one of your lower-tier abilities with a different one from a lower tier.

\begin{itemize}
\item Again and Again
\item Finishing Blow
\item Magnificent Moment
\item Murderer
\item Spin Attack
\item Weapon and Body
\end{itemize}

\subsection{Warrior Example}\index{Types!Warrior!Warrior Example}

Ray wants to create a Warrior character for a modern campaign. He decides that the character is an ex-military fellow who is fast and strong. He puts 3 of his additional points into his Might Pool and 3 into his Speed Pool; his stat Pools are now Might 13, Speed 13, and Intellect 8. As a first-tier character, his Effort is 1, his Might Edge is 1, and his Speed Edge and Intellect Edge are both 0. His character is not particularly smart or charismatic.

He wants to use a large combat knife (a medium weapon that inflicts 4 points of damage) and a .357 Magnum (a heavy pistol that inflicts 6 points of damage but requires the use of both hands). Ray decides not to wear armor, as it’s not really appropriate to the setting, so for his first ability, he chooses Trained Without Armor so he eases Speed defense actions. For his second ability, he chooses Combat Prowess so he can inflict extra damage with his big knife. 

Ray wants to be fast as well as tough, so he selects Improved Edge. This gives him a Speed Edge of 1. He rounds out his character with Physical Skills and chooses swimming and running.

The Warrior can bear two cyphers. The GM decides that Ray’s first cypher is a pill that restores 6 points of Might when swallowed, and his second is a small, easily concealed grenade that explodes like a firebomb when thrown, inflicting 3 points of damage to all within immediate range. 

Ray still needs to choose a descriptor and a focus. Looking ahead to the descriptor rules, Ray chooses Strong, which increases his Might Pool to 17. He also becomes trained in jumping and breaking inanimate objects. (If he had chosen jumping as one of his physical skills, the Strong descriptor would have made him specialized in jumping instead of trained.) Being Strong also gives Ray an extra medium or heavy weapon. He chooses a baseball bat that he’ll use in a pinch. He keeps it in the trunk of his car.

For his focus, Ray chooses Masters Weaponry. This gives him yet another weapon of high quality. He chooses another combat knife and asks the GM if he could use it in his left hand—not to make attacks, but as a shield. This will ease his Speed defense rolls if he has both knives out (the “shield” counts as an asset). The GM agrees. During the game, Ray’s Warrior will be hard to hit—he is trained in Speed defense rolls, and his extra knife eases his defense rolls by another step.

Thanks to his focus, he also inflicts 1 additional point of damage with his chosen weapon. Now he inflicts 6 points of damage with his blade. Ray’s character is a deadly combatant, likely starting the game with a reputation as a knife fighter.

For his character arc, Ray chooses Defeat a Foe. That foe, he decides, is none other than someone in his company who was once a friend but went rogue.

\section{Adept}\index{Types!Adept}
\textbf{Fantasy/Fairy tale}: wizard, mage, sorcerer, cleric, druid, seer, diabolist, fey-touched

\textbf{Modern/Horror/Romance}: psychic, occultist, witch, practitioner, medium, fringe scientist

\textbf{Science fiction}: psion, psionicist, telepath, seeker, master, scanner, ESPer, abomination

\textbf{Superhero/Post-Apocalyptic}: mage, sorcerer, power-wielder, master, psion, telepath

You master powers or abilities outside the experience, understanding, and sometimes belief of others. They might be magic, psychic powers, mutant abilities, or just a wide variety of intricate devices, depending on the setting. (“Magic” here is a term used very loosely. It’s a catch-all for the kinds of wondrous, possibly supernatural things that your character can do that others cannot. It might actually be an expression of technological devices, channeling spirits, mutations, psionics, nanotechnology, or any number of other sources.)

\textbf{Individual Role}: Adepts are usually thoughtful, intelligent types. They often think carefully before acting and rely heavily on their supernatural abilities.

\textbf{Group Role}: Adepts are not powerful in straightforward combat, although they often wield abilities that provide excellent combat support, both offensively and defensively. They sometimes possess abilities that facilitate overcoming challenges. For example, if the group must get through a locked door, an Adept might be able to destroy it or teleport everyone to the other side. 

\textbf{Societal Role}: In settings where the supernatural is rare, strange, or feared, Adepts are likely rare and feared as well. They remain hidden, shadowy figures. When this is not the case, Adepts are more likely to be common and forthright. They might even take leadership roles.

\textbf{Advanced Adepts}: Even at low tiers, Adept powers are impressive. Higher-tier Adepts can accomplish amazing deeds that can reshape matter and the environment around them.

(Adepts are almost always emblematic of the paranormal or superhuman in some way—wizards, psychics, or something similar. If the game you’re playing has none of that, an Adept could be a charlatan mimicking such abilities with tricks and hidden devices, or a gadgeteer character with a “utility belt” full of oddments. Or a game like that might not have Adepts. That’s okay too.)

\subsection{Adept Player Intrusions}\index{Types!Adept!Player Intrusions}

When playing an Adept, you can spend 1 XP to use one of the following player intrusions, provided the situation is appropriate and the GM agrees.

\textbf{Advantageous Malfunction}: A device being used against you malfunctions. It might harm the user or one of their allies for a round, or activate a dramatic and distracting side effect for a few rounds.

\textbf{Convenient Idea:} A flash of insight provides you with a clear answer or suggests a course of action with regard to an urgent question, problem, or obstacle you’re facing.

\textbf{Inexplicably Unbroken}: An inactive, ruined, or presumed-destroyed device temporarily activates and performs a useful function relevant to the situation. This is enough to buy you some time for a better solution, alleviate a complication that was interfering with your abilities, or just get you one more use out of a depleted cypher or artifact.

\subsection{Adept Stat Pools}\index{Types!Adept!Stat Pools}

\begin{table}[H]
\centering
\caption{Adept Stat Pools}
\label{tab: Adept Stat Pools}

\begin{tabularx}{\linewidth}{| X | X |}
% \begin{tabularx}{\linewidth}{| p{0.5\linewidth} | p{0.5\linewidth} |}
\hline
\textbf{Stat} & \textbf{Pool Starting Value} \\
\hline
% Tutaj wpisujesz dane do tabeli
Might & 7 \\ \hline
Speed & 9 \\ \hline
Intelect & 12  \\ \hline

\end{tabularx}

\end{table}

\raggedright

You get 6 additional points to divide among your stat Pools however you wish.

\subsection{Adept Background Connection}\index{Types!Adept!Background Connection}

Your type helps determine the connection you have to the setting. Roll a d20 or choose from the following list to determine a specific fact about your background that provides a connection to the rest of the world. You can also create your own fact.

\begin{table*}
\centering

\caption{Adept Background Connection}
\label{tab:Adept Background Connection}

\begin{tabularx}{\textwidth}{| p{0.05\textwidth} | p{0.9\textwidth} |}
\hline
\textbf{d20} & \textbf{Background} \\
\hline
% Tutaj wpisujesz dane do tabeli
1 & You served as an apprentice for an Adept respected and feared by many people. Now you bear their mark. \\ \hline
2 & You studied in a school infamous for its dark, brooding instructors and graduates. \\ \hline
3 & You learned your abilities in the temple of an obscure god. Its priests and worshippers, although small in number, respect and admire your talents and potential.  \\ \hline
4 & While traveling alone, you saved the life of a powerful person. They remain indebted to you.  \\ \hline
5 & Your mother was a powerful Adept while she lived, helpful to many locals. They look upon you kindly, but they also expect much from you.  \\ \hline
6 & You owe money to a number of people and don’t have the funds to pay your debts. \\ \hline
7 & You failed disgracefully at your initial studies with your teacher and now proceed on your own. \\ \hline
8 & You learned your skills faster than your teachers had ever seen before. The powers that be took notice and are paying close attention.  \\ \hline
9 & You killed a well-known criminal in self-defense, earning the respect of many and the enmity of a dangerous few.  \\ \hline
10 & You trained as a Warrior, but your Adept predilections eventually led you down a different path. Your former comrades don’t understand you, but they respect you.  \\ \hline
11 & While studying to be an Adept, you worked as an assistant for a bank, making friends with the owner and the clientele. \\ \hline
12 & Your family owns a large vineyard nearby known to all for its fine wine and fair business dealings. \\ \hline
13 & You trained for a time with a group of influential Adepts, and they still look upon you with fondness. \\ \hline
14 & You worked the gardens in the palace of an influential noble or person of wealth. They wouldn’t remember you, but you made friends with their young daughter. \\ \hline
15 & An experiment you conducted in the past went horribly awry. The locals remember you as a dangerous and foolhardy individual. \\ \hline
16 & You hail from a distant place where you were well known and regarded, but people here treat you with suspicion. \\ \hline
17 & People you meet seem put off by the strange birthmark on your face. \\ \hline
18 & Your best friend is also an Adept. You and your friend share discoveries and secrets readily. \\ \hline
19 & You know a local merchant very well. Since you give them so much business, they offer you discounts and special treatment. \\ \hline
20 & You belong to a secretive social club that gathers monthly to drink and talk. \\ \hline

\end{tabularx}

\end{table*}

\subsection{First-tier Adept}\index{Types!Adept!1st Tier Adept}

First-tier Adepts have the following abilities:

\textbf{Effort}: Your Effort is 1. 

\textbf{Genius}: You have an Intellect Edge of 1, a Might Edge of 0, and a Speed Edge of 0. 

\textbf{Expert Cypher Use}: You can bear three cyphers at a time. 
Starting Equipment: Appropriate clothing, plus two expensive items, two moderately priced items, and up to four inexpensive items of your choice.

\textbf{Weapons}: You can use light weapons without penalty. You have an inability with medium weapons and heavy weapons; your attacks with medium and heavy weapons are hindered.

\textbf{Special Abilities}: Choose four of the abilities listed below. You can’t choose the same ability more than once unless its description says otherwise. The full description for each listed ability can be found in Abilities, which also has descriptions for flavor and focus abilities in a single vast catalog. (Adept abilities require at least one free hand unless the GM says otherwise.)

\begin{itemize}
\item Distortion
\item Erase Memories 
\item Far Step
\item Hedge Magic
\item Magic Training
\item Onslaught
\item Push
\item Resonance Field
\item Scan
\item Shatter
\item Ward
\end{itemize}

\subsection{Second-tier Adept}\index{Types!Adept!2nd Tier Adept}

Choose one of the abilities listed below (or from a lower tier) to add to your repertoire. In addition, you can replace one of your lower-tier abilities with a different one from a lower tier.

\begin{itemize}
\item Adaptation
\item Cutting Light
\item Hover
\item Mind Reading 
\item Retrieve Memories
\item Reveal
\item Stasis
\end{itemize}

\subsection{Third-tier Adept}\index{Types!Adept!3rd Tier Adept}

Choose two of the abilities listed below (or from a lower tier) to add to your repertoire. In addition, you can replace one of your lower-tier abilities with a different one from a lower tier.

\begin{itemize}
\item Adroit Cypher Use
\item Countermeasures
\item Energy Protection
\item Fire and Ice
\item Force Field Barrier
\item Sensor
\item Targeting Eye
\end{itemize}

\subsection{Fourth-tier Adept}\index{Types!Adept!4th Tier Adept}

Choose one of the abilities listed below (or from a lower tier) to add to your repertoire. In addition, you can replace one of your lower-tier abilities with a different one from a lower tier.

\begin{itemize}
\item Death Touch
\item Exile
\item Invisibility
\item Matter Cloud
\item Mind Control
\item Projection
\item Rapid Processing 
\item Regeneration
\item Reshape
\item Wormhole
\end{itemize}

\subsection{Fifth-tier Adept}\index{Types!Adept!5th Tier Adept}

Choose two of the abilities listed below (or from a lower tier) to add to your repertoire. In addition, you can replace one of your lower-tier abilities with a different one from a lower tier.

\begin{itemize}
\item Absorb Energy
\item Concussion
\item Conjuration
\item Create
\item Dust to Dust 
\item Knowing the Unknown
\item Master Cypher Use
\item Teleportation
\item True Senses
\end{itemize}

\subsection{Sixth-tier Adept}\index{Types!Adept!6th Tier Adept}

Choose one of the abilities listed below (or from a lower tier) to add to your repertoire. In addition, you can replace one of your lower-tier abilities with a different one from a lower tier.

\begin{itemize}
\item Control Weather
\item Earthquake
\item Move Mountains
\item Traverse the Worlds
\item Usurp Cypher
\end{itemize}

\subsection{Adept Example}\index{Types!Adept!Adept Example}

Jen wants to create an Adept—a sorcerer for a fantasy campaign. She decides to be somewhat well rounded, so she puts 2 of her additional points into each stat Pool, giving her a Might Pool of 9, a Speed Pool of 11, and an Intellect Pool of 14. Her Adept is smart and quick. She has an Intellect Edge of 1, a Might Edge of 0, and a Speed Edge of 0. As a first-tier character, her Effort is 1. As her initial abilities, she chooses Onslaught and Ward, giving her a strong offense and defense. She also chooses Magic Training and rounds out her character with Scan, which she hopes will be useful in gaining insight and information. For this character, Onslaught, Ward, and Scan are all spells she has mastered through years of training and study.

She can bear three cyphers. The GM gives her a potion that acts as a short-range teleporter, a small charm that restores 5 points to her Intellect Pool, and a fluid-filled flask that explodes like a fiery bomb. Jen’s sorcerer is skilled with light weapons, so she chooses a dagger. 
For her descriptor, Jen chooses Graceful, which adds 2 points to her Speed Pool, bringing it to 13. That descriptor means she is trained in balancing and anything requiring careful movements, physical performing arts, and Speed defense tasks. Perhaps she is a dancer. In fact, she begins to develop a backstory that involves graceful, lithe movements that she incorporates into her spells.

For her focus, she chooses Leads. This gives her training in social interactions, which again helps round her out—she’s good in all kinds of situations. Moreover, she has the Good Advice ability, which enables her to be a focal point of her group.

Her spells and focus abilities cost Intellect points to activate, so she’s glad to have a lot of points in her Intellect Pool. In addition, her Intellect Edge will help reduce those costs. If she uses her Onslaught force blast without applying Effort, it costs 0 Intellect points and deals 4 points of damage. Her Intellect Edge will allow her to save points to devote toward applying Effort for other purposes, perhaps to boost the accuracy of Onslaught.

For her character arc, Jen chooses Aid a Friend. She decides that when her sorcerer character was young, she had a magical mentor. That mentor was later taken prisoner by a demon, so her character is always looking for clues on how to find the demon and release her friend from bondage.

(GMs are always free to pre-select a type’s special abilities at a given tier to reinforce the setting. In the fantasy setting of Jen’s sorcerer, the GM might have said that all sorcerers (Adepts) start with Magic Training as one of their tier 1 abilities. This doesn’t make the character any less powerful or special, but it says something about her role in the world and expectations in the game.)

\section{Explorer}\index{Types!Explorer}

\textbf{Fantasy/Fairy tale}: Explorer, adventurer, delver, mystery seeker

\textbf{Modern/Horror/Romance}: athlete, explorer, adventurer, drifter, detective, scholar, spelunker, trailblazer, investigative reporter

\textbf{Science fiction}: Explorer, adventurer, wanderer, planetary specialist, xenobiologist

\textbf{Superhero/Post-Apocalyptic}: adventurer, crimefighter

You are a person of action and physical ability, fearlessly facing the unknown. You travel to strange, exotic, and dangerous places, and discover new things. This means you’re physical but also probably knowledgeable.

\textbf{Individual Role}: Although Explorers can be academics or well studied, they are first and foremost interested in action. They face grave dangers and terrible obstacles as a routine part of life. 

\textbf{Group Role}: Explorers sometimes work alone, but far more often they operate in teams with other characters. The Explorer frequently leads the way, blazing the trail. However, they’re also likely to stop and investigate anything intriguing they stumble upon. 

\textbf{Societal Role}: Not all Explorers are out traipsing through the wilderness or poking about an old ruin. Sometimes, an Explorer is a teacher, a scientist, a detective, or an investigative reporter. In any event, an Explorer bravely faces new challenges and gathers knowledge to share with others.

\textbf{Advanced Explorers}: Higher-tier Explorers gain more skills, some combat abilities, and a number of abilities that allow them to deal with danger. In short, they become more and more well-rounded, able to deal with any challenge.

\subsection{Explorer Player Intrusions}\index{Types!Explorer!Player Intrusions}
When playing an Explorer, you can spend 1 XP to use one of the following player intrusions, provided the situation is appropriate and the GM agrees.

\textbf{Fortuitous Malfunction}: A trap or a dangerous device malfunctions before it can affect you.

\textbf{Serendipitous Landmark}: Just when it seems like the path is lost (or you are), a trail marker, a landmark, or simply the way the terrain or corridor bends, rises, or falls away suggests to you the best path forward, at least from this point.

\textbf{Weak Strain}: The poison or disease turns out not to be as debilitating or deadly as it first seemed, and inflicts only half the damage that it would have otherwise.

\subsection{Explorer Stat Pools}\index{Types!Explorer!Stat Pools}

\begin{table}[H]
\centering
\caption{Explorer Stat Pools}
\label{tab: Explorer Stat Pools}

\begin{tabularx}{\linewidth}{| X | X |}
% \begin{tabularx}{\linewidth}{| p{0.5\linewidth} | p{0.5\linewidth} |}
\hline
\textbf{Stat} & \textbf{Pool Starting Value} \\
\hline
% Tutaj wpisujesz dane do tabeli
Might & 10 \\ \hline
Speed & 9 \\ \hline
Intelect & 9  \\ \hline

\end{tabularx}

\end{table}

\raggedright

You get 6 additional points to divide among your stat Pools however you wish.

\subsection{Explorer Background Connection}\index{Types!Explorer!Background Connection}

Your type helps determine the connection you have to the setting. Roll a d20 or choose from the following list to determine a specific fact about your background that provides a connection to the rest of the world. You can also create your own fact.

\begin{table*}
\centering

\caption{Explorer Background Connection}
\label{tab:Explorer Background Connection}

\begin{tabularx}{\textwidth}{| p{0.05\textwidth} | p{0.9\textwidth} |}
\hline
\textbf{d20} & \textbf{Background} \\
\hline
% Tutaj wpisujesz dane do tabeli
1 & You were a star high school athlete. You’re still in great shape, but those were the glory days, man. \\ \hline
2 & Your brother is the lead singer in a really popular band. \\ \hline
3 & You have made a number of discoveries in your explorations, but not all opportunities to capitalize on them have panned out yet.  \\ \hline
4 & You were a cop, but you gave it up after encountering corruption on the force.  \\ \hline
5 & Your parents were missionaries, so you spent much of your young life traveling to exotic places. \\ \hline
6 & You served in the military with honor. \\ \hline
7 & You received assistance from a secretive organization, which paid for your schooling. Now they seem to want a lot more from you. \\ \hline
8 & You went to a prestigious university on an athletic scholarship, but you excelled in class as well as on the field.  \\ \hline
9 & Your best friend from your youth is now an influential member of the government.  \\ \hline
10 & You used to be a teacher. Your students remember you fondly. \\ \hline
11 & You worked as a small-time criminal operative until you were caught and served some time in jail, after which you tried to go straight. \\ \hline
12 & Your greatest discovery to date was stolen by your arch-rival. \\ \hline
13 & You belong to an exclusive organization of Explorers whose existence is not widely known. \\ \hline
14 & You were kidnapped as a small child under mysterious circumstances, although you were recovered safely. The case still has some notoriety. \\ \hline
15 & When you were young, you were addicted to narcotics, and now you are a recovering addict. \\ \hline
16 & While exploring a remote location, you saw something strange you’ve never been able to explain. \\ \hline
17 & You own a small bar or restaurant. \\ \hline
18 & You published a book about some of your exploits and discoveries, and it has achieved some acclaim. \\ \hline
19 & Your sister owns a store and gives you a hefty discount. \\ \hline
20 & Your father is a high-ranking officer in the military with many connections. \\ \hline

\end{tabularx}

\end{table*}

\subsection{First-Tier Explorer}\index{Types!Explorer!1st Tier Explorer}

First-tier Explorers have the following abilities:

\textbf{Effort}: Your Effort is 1.

\textbf{Physical Nature}: You have a Might Edge of 1, a Speed Edge of 0, and an Intellect Edge of 0.

\textbf{Cypher Use}: You can bear two cyphers at a time.
Starting Equipment: Appropriate clothing and a weapon of your choice, plus two expensive items, two moderately priced items, and up to four inexpensive items.

\textbf{Weapons}: You can use light and medium weapons without penalty. You have an inability with heavy weapons; your attacks with heavy weapons are hindered.

\textbf{Special Abilities}: Choose four of the abilities listed below. You can’t choose the same ability more than once unless its description says otherwise. The full description for each listed ability can be found in Abilities, which also has descriptions for flavor and focus abilities in a single vast catalog.

\begin{itemize}
\item Block
\item Danger Sense
\item Decipher
\item Endurance
\item Find the Way
\item Fleet of Foot
\item Improved Edge
\item Knowledge Skills
\item Muscles of Iron
\item No Need for Weapons
\item Physical Skills
\item Practiced in Armor
\item Practiced With All Weapons
\item Surging Confidence
\item Trained Without Armor
\end{itemize}

\subsection{Second-Tier Explorer}\index{Types!Explorer!2nd Tier Explorer}

Choose four of the abilities listed below (or from a lower tier) to add to your repertoire. In addition, you can replace one of your lower-tier abilities with a different one from a lower tier.

\begin{itemize}
\item Curious
\item Danger Instinct
\item Enable Others
\item Escape
\item Eye for Detail 
\item Foil Danger
\item Hand to Eye
\item Investigative Skills
\item Quick Recovery
\item Range Increase
\item Skill With Defense
\item Stand Watch
\item Travel Skills
\item Wreck
\end{itemize}

\subsection{Third-Tier Explorer}\index{Types!Explorer!3rd Tier Explorer}

Choose three of the abilities listed below (or from a lower tier) to add to your repertoire. In addition, you can replace one of your lower-tier abilities with a different one from a lower tier.

\begin{itemize}
\item Controlled Fall
\item Experienced in Armor
\item Expert Cypher Use
\item Ignore the Pain
\item Obstacle Running
\item Resilience
\item Run and Fight
\item Seize the Moment
\item Skill With Attacks
\item Stone Breaker
\item Think Your Way Out
\item Trapfinder
\item Wrest From Chance
\end{itemize}

\subsection{Fourth-Tier Explorer}\index{Types!Explorer!4th Tier Explorer}

Choose two of the abilities listed below (or from a lower tier) to add to your repertoire. In addition, you can replace one of your lower-tier abilities with a different one from a lower tier.

\begin{itemize}
\item Capable Warrior
\item Expert Skill
\item Increased Effects
\item Read the Signs
\item Runner
\item Subtle Steps
\item Tough As Nails
\end{itemize}

\subsection{Fifth-Tier Explorer}\index{Types!Explorer!5th Tier Explorer}

Choose three of the abilities listed below (or from a lower tier) to add to your repertoire. In addition, you can replace one of your lower-tier abilities with a different one from a lower tier.

\begin{itemize}
\item Adroit Cypher Use
\item Free to Move
\item Group Friendship
\item Hard to Kill
\item Jump Attack
\item Mastery With Defense
\item Parry
\item Physically Gifted
\item Take Command
\item Vigilant
\end{itemize}

\subsection{Sixth-Tier Explorer}\index{Types!Explorer!6th Tier Explorer}

Choose three of the abilities listed below (or from a lower tier) to add to your repertoire. In addition, you can replace one of your lower-tier abilities with a different one from a lower tier.

\begin{itemize}
\item Again and Again
\item Inspire Coordinated Actions
\item Mastery in Armor
\item Mastery With Attacks
\item Negate Danger
\item Share Defense
\item Spin Attack
\item Wild Vitality
\end{itemize}

\subsection{Explorer Example}\index{Types!Explorer!Explorer Example}

Sam decides to create an Explorer character for a science fiction campaign. This character will be a hardy soul who explores alien worlds. They put 3 additional points into their Might Pool, 2 into their Speed Pool, and 1 into their Intellect Pool; their stat Pools are now Might 13, Speed 11, and Intellect 10. As a first-tier character, their Effort is 1, their Might Edge is 1, and their Speed Edge and Intellect Edge are 0. Their character is fairly well-rounded so far.

Sam immediately leaps in and starts choosing abilities. They pick Danger Sense and Surging Confidence, thinking those abilities will be generally useful. They also choose Practiced in Armor, reasoning that the character wears high-tech medium armor when exploring. Last, they choose Knowledge Skills and select geology and biology to help during interplanetary explorations. 

Sam’s Explorer can bear two cyphers, which in this setting involve nanotechnology. The GM decides that one is a nanite injector that grants a +1 bonus to Might Edge when used, and the other is a device that can create one simple handheld object the user wishes. 

Sam’s Explorer is not really geared toward fighting, but sometimes the universe is a dangerous place, so they note that they’re carrying a medium blaster as well.
Sam still needs a descriptor and a focus. Looking to the Descriptor chapter, they choose Hardy, which increases their Might Pool to 17. They also heal more quickly and can operate better when injured. They’re trained in Might defense but have an inability with initiative; however, it’s effectively canceled out by their Danger Sense (and vice versa). Sam could go back and select something else instead of Danger Sense, but they like it and decide to keep it. Overall, the descriptor ends up making the character tough but a little slow.

For their focus, Sam chooses Explores Dark Places (in this case, weird ruins of alien civilizations). This gives the character a bunch of additional skills: searching, listening, climbing, balancing, and jumping. They’re quite the capable Explorer.

For their character arc, Sam chooses Enterprise. Exploring alien places sometimes turns up strange relics, and Sam figures they might be able to set up a service to reliably transport these items to responsible third parties, rather than allow them to fall into the hands of pirates and rich private collectors. For a small fee, of course.

\section{Speaker}\index{Types!Speaker}

\textbf{Fantasy/Fairy tale}: bard, speaker, skald, emissary, priest, advocate

\textbf{Modern/Horror/Romance}: diplomat, charmer, face, spinner, manipulator, minister, mediator, lawyer

\textbf{Science fiction}: diplomat, empath, glam, consul, legate

\textbf{Superhero/Post-Apocalyptic}: charmer, mesmerist, puppet master

You’re good with words and good with people. You talk your way past challenges and out of jams, and you get people to do what you want.

\textbf{Individual Role}: Speakers are smart and charismatic. They like people and, more important, they understand them. This helps speakers get others to do what needs to be done. 

\textbf{Group Role}: The Speaker is often the face of the group, serving as the person who speaks for all and negotiates with others. Combat and action are not a Speaker’s strong suits, so other characters sometimes have to defend the Speaker in times of danger.

\textbf{Societal Role}: Speakers are frequently political or religious leaders. Just as often, however, they are con artists or criminals. 
Advanced Speakers: Higher-tier speakers use their abilities to control and manipulate people as well as aid and nurture their friends. They can talk their way out of danger and even use their words as weapons.

\subsection{Speaker Player Intrusions}\index{Types!Speaker!Player Intrusions}

When playing a Speaker, you can spend 1 XP to use one of the following player intrusions, provided the situation is appropriate and the GM agrees.

\textbf{Friendly NPC}: An NPC you don’t know, someone you don’t know that well, or someone you know but who hasn’t been particularly friendly in the past chooses to help you, though doesn’t necessarily explain why. Maybe they’ll ask you for a favor in return afterward, depending on how much trouble they go to.

\textbf{Perfect Suggestion}: A follower or other already-friendly NPC suggests a course of action with regard to an urgent question, problem, or obstacle you’re facing.

\textbf{Unexpected Gift}: An NPC hands you a physical gift you were not expecting, one that helps put the situation at ease if things seem strained, or provides you with a new insight for understanding the context of the situation if there’s something you’re failing to understand or grasp.

\subsection{Speaker Stat Pools}\index{Types!Speaker!Stat Pools}

\begin{table}[H]
\centering
\caption{Speaker Stat Pools}
\label{tab: Speaker Stat Pools}

\begin{tabularx}{\linewidth}{| X | X |}
% \begin{tabularx}{\linewidth}{| p{0.5\linewidth} | p{0.5\linewidth} |}
\hline
\textbf{Stat} & \textbf{Pool Starting Value} \\
\hline
% Tutaj wpisujesz dane do tabeli
Might & 8 \\ \hline
Speed & 9 \\ \hline
Intelect & 11  \\ \hline

\end{tabularx}

\end{table}

\raggedright

You get 6 additional points to divide among your stat Pools however you wish. 

\subsection{Speaker Background Connection}\index{Types!Speaker!Background Connection}

Your type helps determine the connection you have to the setting. Roll a d20 or choose from the following list to determine a specific fact about your background that provides a connection to the rest of the world. You can also create your own fact.

\begin{table*}
\centering

\caption{Speaker Background Connection}
\label{tab:Speaker Background Connection}

\begin{tabularx}{\textwidth}{| p{0.05\textwidth} | p{0.9\textwidth} |}
\hline
\textbf{d20} & \textbf{Background} \\
\hline
% Tutaj wpisujesz dane do tabeli
1 & One of your parents was a famous entertainer in their early years and hoped you would excel in the same medium. \\ \hline
2 & When you were a teenager, one of your siblings went missing and is presumed dead. The shock rent your family, and it’s something you’ve never gotten over. \\ \hline
3 & You were inducted into a secret society that claims to hold and protect esoteric knowledge opposing the forces of evil.  \\ \hline
4 & You lost one of your parents to alcoholism. They may still be alive, but you’d be hard pressed to find forgiveness.  \\ \hline
5 & You have no memory of anything that happened to you before the age of 18. \\ \hline
6 & Your grandparents raised you on a farm far from bustling urban centers. You like to think the instruction they gave you prepared you for anything. \\ \hline
7 & As an orphan, you had a difficult childhood, and your entry into adulthood was challenging. \\ \hline
8 & You grew up in extreme poverty, among criminals. You still have some connections with the old neighborhood.  \\ \hline
9 & You served as an envoy for a powerful and influential person in the past, and they still look upon you with favor.  \\ \hline
10 & You have an annoying rival who always seems to get in your way or foil your plans. \\ \hline
11 & You’ve worked yourself into the position of spokesperson for an organization or company of some importance.\\ \hline
12 & Your neighbors were murdered, and the mystery remains unsolved. \\ \hline
13 & You have traveled extensively, and during that time you accumulated quite a collection of strange souvenirs. \\ \hline
14 & Your childhood sweetheart ended up with your best friend (now your ex-best friend). \\ \hline
15 & You are part of a maligned minority, but you work to bring the injustice of your status to public attention. \\ \hline
16 & You’re part owner of a local bar, where you’re something of a whiz in creating specialty cocktails. \\ \hline
17 & You once ran a con that cheated important people out of money, and they want revenge.  \\ \hline
18 & You used to act in a traveling theater, and they remember you fondly (as do people in the places you visited). \\ \hline
19 & You are in a close romantic relationship with someone in local politics. \\ \hline
20 & Someone out there tries to pose as you, using your identity, often for nefarious ends. You’ve never met the culprit, but you’d certainly like to. \\ \hline

\end{tabularx}

\end{table*}

\subsection{First-Tier Speaker}\index{Types!Speaker!1st Tier Speaker}

First-tier speakers have the following abilities:

\textbf{Effort}: Your Effort is 1.

\textbf{Genius}: You have an Intellect Edge of 1, a Might Edge of 0, and a Speed Edge of 0.

\textbf{Cypher Use}: You can bear two cyphers at a time.

\textbf{Weapons}: You can use light weapons without penalty. You have an inability with medium and heavy weapons; your attacks with medium and heavy weapons are hindered.

\textbf{Starting Equipment}: Appropriate clothing and a light weapon of your choice, plus two expensive items, two moderately priced items, and up to four inexpensive items.

\textbf{Special Abilities}: Choose four of the abilities listed below. You can’t choose the same ability more than once unless its description says otherwise. The full description for each listed ability can be found in Abilities, which also has descriptions for flavor and focus abilities in a single vast catalog. (Some Speaker abilities, like Mind Reading or True Senses, imply a supernatural element. If this is inappropriate to the character or the setting, these abilities can be replaced with something from the stealth flavor, or the GM can slightly modify them so they are based in extraordinary talents and insight rather than the supernatural.)

\begin{itemize}
\item Anecdote
\item Babel
\item Demeanor of Command
\item Encouragement
\item Enthrall
\item Erase Memories
\item Fast Talk
\item Inspire Aggression
\item Interaction Skills
\item Practiced With Medium Weapons 
\item Spin Identity
\item Terrifying Presence
\item Understanding
\end{itemize}

\subsection{Second-Tier Speaker}\index{Types!Speaker!2nd Tier Speaker}

Choose two of the abilities listed below (or from a lower tier) to add to your repertoire. In addition, you can replace one of your lower-tier abilities with a different one from a lower tier.

\begin{itemize}
\item Basic Follower
\item Calm Stranger
\item Disincentivize
\item Gather Intelligence
\item Impart Ideal
\item Inspiring Ease
\item Interaction Skills
\item Practiced in Armor
\item Skill With Defense
\item Speedy Recovery
\item Unexpected Betrayal
\end{itemize}

\subsection{Third-Tier Speaker}\index{Types!Speaker!3rd Tier Speaker}

Choose three of the abilities listed below (or from a lower tier) to add to your repertoire. In addition, you can replace one of your lower-tier abilities with a different one from a lower tier.

\begin{itemize}
\item Accelerate
\item Blend In
\item Discerning Mind
\item Expert Cypher Use
\item Expert Follower
\item Grand Deception
\item Lead by Inquiry
\item Mind Reading
\item Oratory
\item Perfect Stranger
\item Quick Wits
\item Telling
\end{itemize} 

\subsection{Fourth-Tier Speaker}\index{Types!Speaker!4th Tier Speaker}

Choose two of the abilities listed below (or from a lower tier) to add to your repertoire. In addition, you can replace one of your lower-tier abilities with a different one from a lower tier.

\begin{itemize}
\item Anticipate Attack
\item Confounding Banter
\item Feint
\item Heightened Skills
\item Psychosis
\item Read the Signs
\item Spur Effort
\item Strategize
\item Suggestion
\end{itemize}

\subsection{Fifth-Tier Speaker}\index{Types!Speaker!5th Tier Speaker}

Choose three of the abilities listed below (or from a lower tier) to add to your repertoire. In addition, you can replace one of your lower-tier abilities with a different one from a lower tier.

\begin{itemize}
\item Adroit Cypher Use
\item Discipline of Watchfulness
\item Experienced in Armor
\item Flee
\item Foul Aura 
\item Knowing the Unknown
\item Regeneration
\item Skill With Attacks
\item Stimulate
\end{itemize}

\subsection{Sixth-Tier Speaker}\index{Types!Speaker!6th Tier Speaker}

Choose two of the abilities listed below (or from a lower tier) to add to your repertoire. In addition, you can replace one of your lower-tier abilities with a different one from a lower tier.

\begin{itemize} 
\item Assume Control
\item Battle Management
\item Crowd Control
\item Inspiring Success
\item Recruit Deputy
\item Shatter Mind
\item True Senses
\item Word of Command
\end{itemize}

\subsection{Speaker Example}\index{Types!Speaker!Speaker Example}

Mary wants to create a Speaker for a Lovecraftian horror campaign. She puts 3 of her additional stat points into her Intellect Pool and 3 into her Speed Pool; her stat Pools are now Might 8, Speed 12, and Intellect 14. As a first-tier character, her Effort is 1, her Might Edge and Speed Edge are 0, and her Intellect Edge is 1. She’s smart and charismatic but not particularly tough.

Mary chooses Fast Talk and Spin Identity to help get into places and learn things she wants to know. She’s a bit of a con artist. She’s good to her friends, however, and chooses Encouragement as well. Mary rounds out her first-tier abilities with Interaction Skills (deceiving and persuasion).

A Speaker normally starts with two cyphers, but the GM rules that characters in this campaign start with only one—something creepy relating to their background. Mary’s cypher is an odd pocket watch given to her by her grandfather. She doesn’t know how or why, but when activated, the watch allows her to take twice as many actions for three rounds.

Mary’s character carries a small knife hidden in her bag in case of trouble. As a light weapon, it inflicts 2 points of damage, but attacks with it are eased.
Mary chooses Resilient for her descriptor and decides that she can probably learn the truth behind some of the strange things that she’s heard about without feeling too much trauma if it’s horrible. Resilient increases her Might Pool to 10 and her Intellect Pool to 16. She’s trained in Might and Intellect defense actions and gains an extra recovery roll each day. At first, Mary is sad that her descriptor gives her an inability in knowledge and puzzle tasks, but then she realizes that the flaw fits her character well—she’s better at getting people to tell her what she needs to know than at figuring out the information herself.
For her focus, Mary chooses Moves Like a Cat, granting her a final Speed Pool of 18 and training in balance. In the end, she’s graceful and quick, charismatic, and hardier than she initially thought thanks to her drive. She’s ready to investigate the weird.

For her character arc, Mary chooses Fall From Grace. She decides she’s had an obsession with a strange tome that’s been in her family for generations, and her character is drawn to its strange languages and rituals.
