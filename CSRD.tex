\documentclass[a4paper,10pt]{book}
\usepackage[T1]{fontenc}
\usepackage[utf8]{inputenc}
\usepackage{lmodern}

\usepackage{tabularx}
\usepackage{multicol}
\usepackage{fontspec}
\usepackage{imakeidx}
\usepackage{hyperref}
\usepackage{xcolor}
\setlength{\columnseprule}{1pt}

% pakiet babel lub jego odpowiednik

\title{CYPHER SYSTEM REFERENCE DOCUMENT 2025-05-05 final}
\author{Monte Cook Games (source material) \and Szymon "Kaworu" Brycki (LaTeX version)}
\date{\today}

\setmainfont{FreeSerif}[
  UprightFont = FreeSerif,          % Normalny styl (regular)
  BoldFont = FreeSerifBold,         % Pogrubiony styl (bold)
  ItalicFont = FreeSerifItalic,     % Kursywa (italic)
  BoldItalicFont = FreeSerifBoldItalic, % Pogrubiona kursywa (bold italic)
  ExternalLocation, 
  Path=, 
  Extension=.ttf,
]

\hypersetup{
    colorlinks=true,
    linkcolor=black,
    filecolor=black,      
    urlcolor=black,
}

\makeindex

\begin{document}

\maketitle
\tableofcontents

\part{Core Rules}

\chapter{Introduction}

\begin{multicols}{2}

\section{How to Play the Cypher System}
The rules of the Cypher System are quite straightforward at their heart, as all of gameplay is based around a few core concepts.

This chapter provides a brief explanation of how to play the game, and it’s useful for learning the game. Once you understand the basic concepts, you’ll likely want to reference Rules of the Game for a more in-depth treatment. 

The Cypher System uses a twenty-sided die (1d20) to determine the results of most actions. Whenever a roll of any kind is called for and no die is specified, roll a d20.

The game master sets a difficulty for any given task. There are ten degrees of difficulty. Thus, the difficulty of a task can be rated on a scale of 1 to 10.

Each difficulty has a target number associated with it. The target number is always three times the task’s difficulty, so a difficulty 1 task has a target number of 3, but a difficulty 4 task has a target number of 12. To succeed at the task, you must roll the target number or higher. See the Task Difficulty table for guidance in how this works.

Character skills, favorable circumstances, or excellent equipment can decrease the difficulty of a task. For example, if a character is trained in climbing, they turn a difficulty 6 climb into a difficulty 5 climb. This is called easing the difficulty by one step (or just easing the difficulty, which assumes it’s eased by one step). If they are specialized in climbing, they turn a difficulty 6 climb into a difficulty 4 climb. This is called easing the difficulty by two steps. Decreasing the difficulty of a task can also be called easing a task. Some situations increase, or hinder, the difficulty of a task. If a task is hindered, it increases the difficulty by one step.

A skill is a category of knowledge, ability, or activity relating to a task, such as climbing, geography, or persuasiveness. A character who has a skill is better at completing related tasks than a character who lacks the skill. A character’s level of skill is either trained (reasonably skilled) or specialized (very skilled).
If you are trained in a skill relating to a task, you ease the difficulty of that task by one step. If you are specialized, you ease the difficulty by two steps. A skill can never decrease a task’s difficulty by more than two steps.

Anything else that reduces difficulty (help from an ally, a particular piece of equipment, or some other advantage) is referred to as an asset. Assets can never decrease a task’s difficulty by more than two steps.

You can also decrease the difficulty of a given task by applying Effort. (Effort is described in more detail in the Rules of the Game chapter.) 
To sum up, three things can decrease a task’s difficulty: skills, assets, and Effort.

If you can ease a task so its difficulty is reduced to 0, you automatically succeed and don’t need to make a roll.

\subsection{When Do You Roll?}

Any time your character attempts a task, the GM assigns a difficulty to that task, and you roll a d20 against the associated target number.

When you jump from a burning vehicle, swing an axe at a mutant beast, swim across a raging river, identify a strange device, convince a merchant to give you a lower price, craft an object, use a power to control a foe’s mind, or use a blaster rifle to carve a hole in a wall, you make a d20 roll.

However, if you attempt something that has a difficulty of 0, no roll is needed—you automatically succeed. Many actions have a difficulty of 0. Examples include walking across the room and opening a door, using a special ability to negate gravity so you can fly, using an ability to protect your friend from radiation, or activating a device (that you already understand) to erect a force field. These are all routine actions and don’t require rolls.

Using skill, assets, and Effort, you can ease the difficulty of potentially any task to 0 and thus negate the need for a roll. Walking across a narrow wooden beam is tricky for most people, but for an experienced gymnast, it’s routine. You can even ease the difficulty of an attack on a foe to 0 and succeed without rolling.

If there’s no roll, there’s no chance for failure. However, there’s also no chance for remarkable success (in the Cypher System, that usually means rolling a 19 or 20, which are called special rolls; the Rules of the Game chapter also discusses special rolls).

\subsection{Task Difficulty}

\end{multicols}

\begin{table}
\centering

\caption{Task Difficulty}
\label{tab:Task Difficulty}

\begin{tabularx}{\textwidth}{| p{0.15\textwidth} | p{0.3\textwidth} | p{0.15\textwidth} | p{0.3\textwidth} |}
\hline
\textbf{Task \newline Difficulty} & \textbf{Description} & \textbf{Target Number} & \textbf{Guidance} \\
\hline
% Tutaj wpisujesz dane do tabeli
0 & Routine & 0 & Anyone can do this basically every time.  \\ \hline
1 & Simple & 3 & Most people can do this most of the time.  \\ \hline
2 & Standard & 6 & Typical task requiring focus, but most people can usually do this.  \\ \hline
3 & Demanding & 9 & Requires full attention; most people have a 50/50 chance to succeed.  \\ \hline
4 & Difficult & 12 & Trained people have a 50/50 chance to succeed.  \\ \hline
5 & Challenging & 15 & Even trained people often fail.  \\ \hline
6 & Intimidating & 18 & Normal people almost never succeed.  \\ \hline
7 & Formidable & 21 & Impossible without skills or great effort.  \\ \hline
8 & Heroic & 24 & A task worthy of tales told for years afterward.  \\ \hline
9 & Immortal & 27 & A task worthy of legends that last lifetimes.  \\ \hline
10 & Impossible & 30 & A task that normal humans couldn’t consider (but one that doesn’t break the laws of physics).  \\ \hline

\end{tabularx}

\end{table}

\begin{multicols}{2}

\end{multicols}

\listoftables
\printindex
\end{document}
